%headers{{{
\documentclass[a4paper,11pt,openright,twocolumn]{book}
\usepackage{latexsym}
\usepackage[activeacute,spanish]{babel}
\usepackage[utf8]{inputenc}
\pagestyle{plain}
\pretolerance=2000
\tolerance=3000
\author{Aleix Alva}
\title{Eructos filosóficos}
\date{2009 - ¿? \\
\vspace {2cm} (Actualizado el \today)}
\frenchspacing
\setcounter{secnumdepth}{4} 
\usepackage{titlesec}
\titleformat{\section}[display] {\normalfont\huge\bfseries\centering}{\sectiontitlename\ \thesection}{15pt}{\Large}
\begin{document}
\maketitle
%\newpage
%\cleardoublepage
%\vspace{12 pt}
%\begin{flushright}
% \begin{quote}
% \it
%\rightline{Dedicatoria}
%\end{quote}
%\end{flushright}
%\newpage
%\cleardoublepage

%\begin{quote}
%\it \begin{flushright}
% Cita\\ Cita\\
% \end{flushright} 
%\end{quote} 
%\begin{flushright} 
%\hfill{\small autor cita, 
%\emph{Libro cita}} 
%\end{flushright} 
%\newpage 
%\cleardoublepage

%headers}}}


%-----001-----{{{
\section*{001}

El arte como eructo. La filosofía como regüeldo. La creación como el ascenso de un espíritu ligero que quiere liberarse de las entrañas para salir al mundo expandiéndose a la velocidad del sonido. No más concepción, gestación y alumbramiento. Las obras no son hijos, ni falta que hace. Bastante tiene uno ya con la idiotina segregada por los progenitores biológicos como para tener que soportar también una afectación maternal en la creación artística o filosófica. La triste alegoría del retoño no me sirve. Necesito algo que sepa esquivar la estupidez de la trascendencia hormonal y la candidez de la preñez en el arte. ¿Y qué decir de las náuseas sofísticas? Verdaderamente vomitivas... Propongo cambiar de metáfora, construir una más abrupta y artificiosa. Una imagen compatible con la volatilidad de la idea y del instante, que pueda provocar y ofender y a la vez ser una señal de buena educación sin haber contradicción en ello. El arte y la filosofía como disciplinas gaseosas. ¿Qué hay más elevado que esto? Una metáfora tan breve que no permita ningún sistema, que pueda ser profunda y fanfarrona, bravucona y sutil, y tan gravemente aguda como agudamente grave... Los eructos como destellos de buen humor y mejor disgestión. Filosofía y arte como una resonancia seguida de una esencia...


%-----001-----}}}
%-----002-----{{{
\section*{002}

Ante una libreta en blanco solo hay dos opciones dignas: dejarla impoluta o mancharla hasta el final de la mejor forma 
posible. Ambas elecciones son legítimas. Quedarse en la salud nívea o adentrarse en una sombría enfermedad, tal es la distancia que separa el lector del escritor. A pesar de eso, la mayoría de las libretas sufre el mismo final que las tortugas de mar recién nacidas: perecen antes de zarpar o nada más zambullirse. Pero cuando resulta afortunada y es adquirida por una pluma con estrella, se convierte en la unidad literaria por excelencia. Arbitraria y acotada como el propio destino pero dócil e infinita en sus posibilidades. Ante ella se abre todo un mar de riquezas y no pocos peligros. Lejos de ser simples elementos pasivos e inocuos, las libretas se comportan como espejitos mágicos. Alimentadoras de esperanzas, multiplicadoras de ideas, concretadoras de sueños, psicólogas de bajo coste y como consecuencia de todo ello, constatadoras de mediocridad.

%-----002-----}}}
%-----003-----{{{
\section*{003}

Hacer algo que ya está hecho no va en detrimento ni del que lo hizo ni del que lo (re)hace, sino del que no sabe valorar 
la reinvención. La originalidad no la vas a encontrar en la fractalidad irrepetible, ya que esta se genera por repetición. No se trata de que tengas ocurrencias derivadas de una recurrencia, sino que recurras a lo primitivo. Lo que es primordial o troncal necesita savia que circule por ello. La madera que ha quedado de los grandes pensadores tiene una función estructural, pero habrá sido en vano si no recorres esos caminos con la xilema de tu juventud. Redescubrir es la única y verdadera forma de comprender, y esta misma frase es un ejemplo de ello, pues te puede sonar sólida o hueca según tu propio relleno. Si lo que quieres es prestigio, fama, publicaciones o simplemente currículum no pasa nada si te vas por las ramas o por las raíces y no vuelves. Pero si quieres entender, entonces redescubre, reinventa, rehaz... Aprender es el primer paso, el do en la escala que sube al conocimiento. Pero el segundo escalón se llama re y debes pasar por él si aspiras a un si.

%-----003-----}}}
%-----004-----{{{
\section*{004}

Con frecuencia se recurre a una música enlatada para elevar un acto, pero con más frecuencia todavía la capacidad para
 mantenerse elevado de los asistentes es tan baja que se recurre a cortar la grabación mucho antes de que esta llegue a su final natural. A veces de forma abrupta. Otras, con un fade out espantoso. Pero en todo los casos se acaba consiguiendo el efecto contrario al deseado, pues tras el corte la escena queda muy por debajo de lo que estaba antes de empezar la música. El silencio llega con un chillido estrepitoso y deja el acto que quería elevarse en una agonía sonora similar a la de una alcantarilla repleta de ratas ultrasónicas. Es la música, la buena música, la que se venga así de aquellos que la usan con prisas y la amputan liberando un silencio ensordecedor imposible de remontar. ¡No al aborto musical!

%-----004-----}}}
%-----005-----{{{
\section*{005}
Los que buscan la inteligencia artificial han empezado la casa por la ventana. Imaginemos una máquina supuestamente inteligente a la que empezamos a hacer preguntas {\it alla Turing}. La cuestión sobre la primera cuestión sería: ¿por qué la máquina se iba a dignar a contestar? Y en el caso de que lo hiciera, podríamos responder preguntándole por qué ha contestado. A esto último, si {\it sólo}   fuera inteligente, ya no podría responder de forma satisfactoria. Nunca enunciaría algo como: {\it  Contesto
no por ser inteligente sino por estar viva. Responder a tu pregunta es en cierto modo una forma de estupidez.}   ¿Inteligencia artificial? ¿Antes de comprender la vida en la que esta se debe sustentar? La I.A. es una todavía un rebuzno.


%-----005-----}}}
%-----006-----{{{
\section*{006}

Inocencia, capacidad, inteligencia y de nuevo inocencia. Si te quedas en la inteligencia habrás sido tonto. Si te estancas en la capacidad es que no has podido. Si no sales de la inocencia no serás más que un pobre desgraciado, pero si vuelves a ella, si consigues completar el ciclo, y sólo si lo consigues, podrás hacer honor al nombre de tu especie, y quizás fundar una nueva subespecie superior, la del {\it  Homo innocens innocens}  .

%-----006-----}}}
%-----007-----{{{
\section*{007}

El noble arte de las variaciones demuestra que la perfección no es única, o al menos que para serlo debe desarrollar todas
sus posibilidades.

%-----007-----}}}
%-----008-----{{{
\section*{008}

Sea M el conjunto de los mediocres. Sea R el subconjunto de M en el que se encuentran los mediocres que reconocen serlo, es 
decir, que para que un individuo pertenezca a R debe ser mediocre y además reconocer que lo es. Esto último no es suficiente para sacar a R de M pero sí para que M se sienta fracturado por R. No obstante, el consuelo que encuentran los miembros de R en semejante quebrantamiento los reinstaura en el seno de M sin fisuras esta vez. Tal es el castigo reparador y el reparo castigador para los que han buscado en la lógica una fuente de creatividad.

%-----008-----}}}
%-----009-----{{{
\section*{009}

Mi odio a la humanidad se funda en las armas. No tanto por su capacidad de matar como por la forma en la que esta supera
en exceso la fuerza individual de quien la utiliza. No hay nada odioso en quien hinca sus colmillos porque expone a la vez su yugular y sus ojos, pero que a un necio atrofiado le baste accionar un suave gatillo para abatir a la criatura más exhuberante sin hacer peligrar ni una de sus pestañas es algo que merece todo mi desprecio. El arma subyuga una fuerza que no sale del propio individuo. Es preciso distinguir entre la fortaleza que uno mismo posee y la que obtiene reconduciendo un mecanismo externo. El Homo sapiens ha sabido extraer un desarrollo del momento externo cada vez más grande con el precio de un empequeñecimiento de su espíritu. ¿Acaso se podría definir este último como la debilidad expuesta durante la exhibición de la fuerza? De la misma forma que el dinero, un arma puede poseerla (incluso imprimirla) cualquiera. Y así hemos pasado de la ley del más fuerte a la ley de la fuerza, donde es esta el parámetro que se va optimizando y no los individuos que la poseen. Del esfuerzo noble a la atrofia espiritual. Del crecimiento armónico a la divergencia cobarde. Cierto que reclamar una vuelta al uso del vigor de tus propios brazos resulta prehistórico, pero eso no quita que no podamos ver una moral y una estética del poder que impulse a algunos individuos a querer estar a la altura interna de su propia capacidad externa. Ya que no vamos a retroceder en una cosa, ¿por qué no intentar avanzar en la otra? El camino de la explotación del potencial gigantesco que esconde la naturaleza combinado con la necedad ilimitada de la humanidad puede llevar a la autodestrucción, y peor aún, al fin de la Tierra y la mayoría de sus especies. ¿Y no es esa la enorme debilidad que enseñamos cada vez más? Según una estética y una moral basadas en el honor y la nobleza, la evolución no debe tener como límite un poder infinito y una voluntad nula, sino una voluntad de poder tan grande como equilibrada. 

%-----009-----}}}
%-----010-----{{{
\section*{010}

Si, como muchos dan por supuesto, la madurez la da el conocimiento de la muerte, ¿por qué cualquier fiera parece mucho más
madura que casi cualquier persona?  Aunque en última instancia dicha madurez la causara la muerte, esta no se persona hasta el final, pero sí que envía constantemente mensajeros en forma de dolor. El intelectual ve su muerte en la distancia y por eso su madurez aún se permite ser parlanchina. Se trata de una madurez de salón, relajada, de largo alcance.  En otras criaturas, la muerte se huele y se toca en todo momento, y por eso su madurez, mucho más madura, no se puede dar el lujo de la verbalización. Madurez es quizás el silencio de una criatura polarizada por el sufrimiento. Un silencio que le permite no malgastar vida por la boca.

%-----010-----}}}
%-----011-----{{{
\section*{011}

Si yo tuviera que gobernar un país sólo podría ser dictador. Es la única opción política que me parece interesante y seguramente la única capaz de llevar al hombre a un escalón superior, o al menos sacarlo del pozo en que se encuentra. Pero que nadie se confunda: yo, a diferencia de otros, sería un {\it  buen }   dictador, que no es lo mismo que un dictador bueno, claro está. Hoy, la democracia goza de tan buena salud que debe habérsela arrebatado a cada uno de los votantes. Hasta ahora se ha tiranizado a la tiranía más por la mediocridad de los que dictaron que por su autoridad.  Perseguir a negros, judíos, homosexuales... Militarizar cada aspecto de la vida... Hacer apología de la raza pura... Todo eso no son más que estupideces, absurdos que revelan el analfabetismo que ha mandado hasta ahora en la historia dictatorial. Yo sería un dictador moderno, algo que el mundo no ha visto todavía. Un tirano que buscaría lo mejor para todos en función de lo que cada uno valiera sin excluir razas, gustos ni otras especies. Y sobre todo, impondría la vida a todos los demócratas que cada día hacen un holocausto con ella.

%-----011-----}}}
%-----012-----{{{
\section*{012}

El que se pregunta si existe Dios o no debería preguntarse más bien sobre la legitimidad de tal pregunta. De la misma forma que cuestionarse sobre lo que hubo antes del Big Bang es no entender lo que significa esa singularidad, el que se cuestiona la existencia de Dios ignora el significado de la existencia. En la explosión primordial de la causalidad podemos intuir un vértice singular al que Spinoza llamó Dios, pero Dios es un concepto y como tal necesita la existencia como sustento. ¿Y cómo admitir que Dios necesita sustento? ¿Acaso la existencia creó a Dios? Eso llevaría al absurdo lo que entendemos por tal deidad. Dios no puede ser algo subyacente. Ni siquiera puede ser {\it algo}  , pues eso ya equivale a ser un subordinado de la existencia, la cual a la vez depende del propio tiempo. Nuestro conocimiento cuando nos acercamos a las singularidades primordiales empieza a tambalearse y ni siquiera las preguntas que uno se hace mantienen su dignidad. Quien quiera llamar Dios al auténtico origen de la explosión debe cuestionarse que tal cosa pueda ser un concepto. Quizás ni siquiera deba estar ligado a una palabra. Quizás el lenguaje, después de todo, solo sea válido a bajas energías...

%-----012-----}}}
%-----013-----{{{
\section*{013}

Todavía espero al poeta solar, aquel que deje de cantar a la Luna como si esta perteneciera exclusivamente a la noche. Aquel que sea capaz de renunciar a las facilidades de las sombras crepusculares y afrontar la aridez de la sombra mínima. Aquel que dedique sus versos a la salud y a la fuerza, a la mañana y al músculo, y se deje de enfermedades mortecinas y taciturnas. Aquel que cante al amor sin caer en lo venéreo. Aquel que arrebate la exclusividad del lirismo a los poetas afeminados. Aquel que cante a la vida, previo entendimiento de lo que esta significa. Espero a aquel cuyos versos estén pensados con sangre y escritos con tinta. A los demás, les recomiendo alzar su mirada al mediodía y mirar al Sol hasta quedarse ciegos. Entonces, leeré encantado lo que puedan entonar desde su noche sempiterna...

%-----013-----}}}
%-----014-----{{{
\section*{014}

Cuando alguien se escandaliza al ver una caca en la calle, no siempre canina, luciendo su flamante contraste con el resto del suelo, debería reflexionar un poco antes de dar rienda suelta a su indignación. La mierda no es algo deplorable, sino todo lo contrario. Con el excremento se completa un ciclo de reciclaje perfecto en el que la tierra se fertiliza y se prepara para hacer germinar la vida con más fuerza. Es más, la reproducción no es más que una forma especial de excreción. ¿Dónde está lo asqueroso? La verdadera mierda es lo que se interpone entre el zurullo y la tierra capaz de asimilarlo en su seno a través de una compleja y maravillosa serie de rutas fisicoquímicas. Es el asfalto, el mármol o el cemento lo que debería llevarnos al rostro de desprecio. Entre el último eslabón y el primero de una cadena flexible, rica y llena de vida interponemos una capa dura, insultantemente simple y hecha de los mismos materiales con los que sellamos los nichos. Merece pues todo mi asco el que siente el asco ante un excremento en vez de sentirlo por una tumba cada vez más gigante que como todas sus pequeñas homólogas impide la verdadera consumación, que no es la del descanso eterno sino la de la reutilización inteligente.

%-----014-----}}}
%-----015-----{{{
\section*{015}

¿En qué consiste la madurez? En verlo todo cada vez más crudo.

%-----015-----}}}
%-----016-----{{{
\section*{016}

¿Hilo musical? ¡No! ¡Música que pende de un hilo!

%-----016-----}}}
%-----017-----{{{
\section*{017}

No es lo mismo borrar que olvidar. Se borran los trazos de tiza en las pizarras buenas y los recuerdos en las mentes malas. Olvidar involucra algo diferente a una eliminación de datos. El tiempo, maestro del olvido, no borra los recuerdos: los {\it  sumerge}  . Lo hace en una oscura y densa gelatina contenida en un suave y gigantesco embudo en el que la caída de los recuerdos es lenta, sin aceleración. El que es profundo y bucea en sus propios pozos se encuentra con unas pasiones cada vez más abisales. Desde allí ya no es posible evocar una alegría sin revivir amargamente diez penas, ni tampoco recordar una tristeza sin sentir el sabor descarado de los gozos que se ahogaron con ella. Olvidar, por lo tanto, no solo no consiste en borrar los recuerdos sino que trata del almacenamiento indisociable de todos ellos. A veces parece que se borran, es cierto, pero es solo porque no sabemos o no nos atrevemos a descender tanto. Tememos, nosotros los sepultureros gelatinosos, que posean todavía una vida zombi y latente que se despierte con nuestra presencia y nos haga sentir incómodos al ver cómo seguimos nuestras vidas con tanta impasibilidad. Solo los sueños nos recuerdan a veces lo que es el olvido. Por eso los borramos rápidamente. 

%-----017-----}}}
%-----018-----{{{
\section*{018}

Hay algo extraño en el concepto de biografía. Por mucho o muy bien que se hable de la vida y de la obra de alguien, siempre se nos escapará el latido que impulsó aquella vida y aquella obra. Si algún biógrafo consiguiera llegar a tanta profundidad dejaría de ser tal y deberíamos llamarlo artista. Probablemente un talento semejante no estaría interesado en hurgar en la vida de otro sino en explotar la suya propia. El artista que escribe su propia autobiografía, en cambio, solo da síntomas de esterilidad, y ni siquiera él podrá ya alcanzar el latido que una vez llevó dentro. ¿Cuál es el motivo para escribir una biografía? ¿No es acaso un parasitismo de individuos beta que a veces son despojos de un antiguo alfa? ¿Cuál es, no obstante, el motivo para leer biografías? Aquí está la trampa del biógrafo. Con su obra, te hará creer que podrás acercarte al genio, que podrás sentir sus impulsos y tal vez contaminarte con su grandeza. Pero no te engañes: tú, que querías ser sobrehumano, acabas leyendo obras de oportunistas recalcando sin remedio aspectos humanos en los que de verdad superaron ese umbral, o peor, de sobrehumanos que volvieron forzadamente a la modestia. Grafía, como su hermana logía: autopsias de la vida misma.  

%-----018-----}}}
%-----019-----{{{
\section*{019}

La complejidad se puede buscar. La sencillez solo se consigue.

%-----019-----}}}
%-----020-----{{{
\section*{020}

En la ignorancia lo que más destaca es la ausencia de consciencia de esta. Por eso su imagen habitual es tan cómoda como borrosa. A los genes les ha costado menos convertirse en estúpidos miopes de su estupidez que en lúcidos conscientes de su lucidez. ¿No tendrá la evolución como objetivo final una ceguera perfecta, acompañada de otra ceguera, también perfecta, que le impida ver tal perfección? Una oscuridad al cuadrado, un silencio sin paredes. Dos ojos que se miran y dos oídos que se escuchan, pero en un silencio negro. El conocimiento opera con autonomía celoso de su intimidad. La ignorancia se renormaliza autocomplaciente mientras el universo queda en la habitación de al lado, desnudándose impunemente sin que nadie pueda robarle una mirada. Ignoramos. Ignoramos que ignoramos. Ignoramos ignorar lo que ignoramos. Ignoramos que ignorar es la ignorancia de la ignorancia. Ignoramos que al ignorar ignoramos lo que la ignorancia ignora. Diría incluso que ignoramos la ignorancia de ignorar lo que ignorantes ignoramos de ella. Y así jugamos entre sus paredes móviles, ajenos a las luces y a las músicas, ajenos al conocimiento, y al conocimiento del conocimiento. Al conocimiento que se sabe conocedor del conocimiento que le queda por conocer. Incluso diría que ajenos al conocer que reconoce conocer el reconocimiento de su conocido conocimiento. Reconocer que reconocer se reconoce en sí mismo, como una metametáfora de su esencia. Ignorar el conocimiento. Conocer la ignorancia. Ignorar a la rancia de la ignorancia. Reconocer el cimiento del conocimiento. 

%-----020-----}}}
%-----021-----{{{
\section*{021}

¿Qué quiere decir {\it ser}   inteligente? ¿Vivir toda la vida sin decir una sola tontería, sin hacer ninguna torpeza, por pequeña que sea, aprenderlo todo a la perfección a la primera y no equivocarse jamás? Eso es obviamente una estupidez. Otra cosa muy diferente es serlo a ratos, mostrarlo en algunos instantes, o incluso tener esas chispas brillantes más a menudo que otros. Pero eso no nos hace inteligentes, pues también hay momentos para memeces y mediocridades, no menos abundantes en genios que en el resto. ¿Acaso la lucidez o estupidez de un momento debe etiquetarnos de por vida? ¿Por qué no hablamos de acciones en cada individuo y las etiquetamos a ellas en vez de a su progenitor? Una acción surgida de un momento de inteligencia sí que será, probablemente, una acción inteligente para siempre. Y una tontería siempre podrá mantener su condición. ¿Acaso a un creador de una obra maestra que al día siguiente dice la mayor gilipollez nunca oída se le debe promediar, quedándose con un término medio? Son las acciones, los hechos que una persona realiza, los que {\it son}   de una forma u otra. La persona, si se quiere etiquetar, debe ser medida por su histograma, su distribución de acciones. En este sentido, ¿qué es ser un genio? ¿Tener un promedio alto, una sola acción muy elevada, ...? ¿Y qué es ser idiota? ¿Hacer una gran estupidez, tener un pico en zonas de baja puntuación? ¿Cómo llamamos a quien tiene una distribución multimodal, con picos extremos a uno y a otro lado? ¿Y a otra con un pico totalmente centrado? ¿Es más inteligente alguien con un promedio elevado que otro con una sola acción muy brillante? ¿Es la mediocridad lo que está en el centro o lo que queda en la zona negativa? ¿Debemos representar las acciones en una sola dimensión? ¿Puede una acción mostrar e incluso desarrollar una inteligencia superior a la que tenía cuando fue creada? ¿Es inteligente una acción que lo parece en el momento pero que es todo lo contrario a largo plazo? Y finalmente, cuando una idea inteligente destruye a su creador, ¿cómo puntúa aquella en el histograma de este? ¿Como obra maestra y tontería sublime a la vez? La inteligencia como atributo personal no es más que eso, una tenencia gilí...

%-----021-----}}}
%-----022-----{{{
\section*{022}

Hubo un tiempo para Dios. Hubo un tiempo incluso para la muerte de Dios. Pero hoy, mientras para muchos sigue vivo y para otros sigue muriendo, para mí ya es pura ceniza. Año tras año observo las procesiones de Semana Santa y pienso que Jesús ha muerto de tanto morir, por sobredosis de resurrección. De pequeño sentí una mezcla de burla y temor hacia lo religioso. Luego, en la adolescencia, me vino cierto respeto por un tema que ha condicionado seria (y grotescamente) la Historia. Llegué incluso a sentir una atracción pasajera por la estética de esa mística. Finalmente me he cansado de intentar ver algo interesante en ello. Este ha sido mi vía crucis hacia la salida del Calvario, mi crucificación de una Cruz que nunca ha dado la cara, mi cruzada contra la Pasión que tan cara nos ha salido. Quizás nunca pueda librarme de este pomposo vocabulario... pero siento que Dios, más que muerto, está {\it acabado}  . Su último recurso puede que sea la fertilidad de sus cenizas...

%-----022-----}}}
%-----023-----{{{
\section*{023}

Un artista serio debe vivir entre cadenas. Primero crecer encadenado por buenos maestros herreros hasta que, quedándose pequeñas, tenga que forjar y someterse a otras más densas y más pesadas. Un artista libre de cadenas, que se libra de las libras del rigor, no es más que un esclavo de lo liviano. Un artista encadenado por las cargas más exigentes, en cambio, es libre por definición y eso le permite crear obras de peso.

%-----023-----}}}
%-----024-----{{{
\section*{024}

Los que hablan de los animales en tercera persona, desmarcándose de ellos, no incurren en ninguna animalidad. Tienen toda la razón en creerse diferentes, pues los animales, como su nombre indica, son aquellos seres con ánima, es decir, con alma. 

%-----024-----}}}
%-----025-----{{{
\section*{025}

Quien tiene dos solitarias no sufre menos que el que posee solamente una. De la misma forma, cuando se juntan dos soledades, estas tampoco se aniquilan mutuamente. Al revés: compiten potenciando un hambre común a la que llaman compañía.

%-----025-----}}}
%-----026-----{{{
\section*{026}

A menudo intentamos evitar recordar a los muertos para evitar evocar el dolor de su pérdida, pero ¿qué hay de todo lo bueno que nos dieron y que aún pueden darnos como memorias vivas? La muerte ya tiene bastante con ser prácticamente irreversible, no le demos también el poder de la opacidad. Saquemos a los fantasmas de la oscuridad e invitémolos a entrar de nuevo en nosotros, pues ¿a qué tanto llanto cuando tuvieron que salir de nuestra vida si ahora les cerramos la puerta a cal y canto cuando llaman a ella? El luto debe limitarse estrictamente a sus límites fisiológicos. Cuando este pasa no basta con guardar la ropa negra y volver a vestir otros colores. Es necesario también permitir que tomen el sol aquellos que todavía viven en nuestro interior, para que recobren todos sus matices. Mientras avanza el tiempo por fuera hay que aprender a retroceder en él por dentro. De lo contrario, muertos y vivos nos quedamos para siempre estancados en el peor momento, en un punto singular que no es más que una puerta de transición entre dos mundos. Es un gran error enterrar o incinerar los recuerdos junto al cadáver y dejar que la puerta se vuelva opaca. Cierto que no disponemos de la llave para volver a abrirla, pero sí tenemos la capacidad de mantenerla limpia y transparente.


%-----026-----}}}
%-----027-----{{{
\section*{027}

Lo artificial es un concepto ficticio, un artificio de la naturaleza. Lo natural también es un concepto, y por tanto un producto artificial generado naturalmente. La distición entre natural y artificial es algo artificial, aunque también una ficción de lo más natural.

%-----027-----}}}
%-----028-----{{{
\section*{028}

Un jurado popular es injusto por definición, pues su tarea no consiste en restablecer el equilibrio en una balanza en la que un plato está irremediablemente sobrecargado, sino dilucidar a qué bandeja pertenece el acusado. Si, usando la intrínsecamente injusta lógica popular, determina que es inocente, será forzado a reinsertarse, es decir, a engrosar la bandeja masificada, aumentando todavía más el desnivel. Si dictamina que es culpable, sentenciará al individuo a la oscuridad de la celda {\it para compensar}   el hecho de que goce de un plato espacioso, ligero y situado a gran altura.

%-----028-----}}}
%-----029-----{{{
\section*{029}

Solemos creer que es el alma la que vive dentro del cuerpo, porque a este lo vemos y palpamos, mientras que a aquella solo la intuimos. No obstante, cuando hablamos de belleza, vemos cómo la exterior se explica por una fisiología interna y por una genética nuclear, mientras que la interior se forja a través de nuestra interacción con los demás. Y claro, si la belleza interior resulta ser una propiedad externa y la exterior una cualidad interna, ¿no será que es el cuerpo el que vive dentro del alma?

%-----029-----}}}
%-----030-----{{{
\section*{030}

A veces siento que estoy en el mundo para demostrar que nadie, ni siquiera yo, está en el mundo para algo.

%-----030-----}}}
%-----031-----{{{
\section*{031}

La juventud es una enfermedad a la que llega la cura. La vejez es una enfermedad a la que llega un cura. 

%-----031-----}}}
%-----032-----{{{
\section*{032}

Hay grandes obras que, saciando la sed, te acaban dejando seco. Otras en cambio, siendo mediocres y faltas de jugo, dan una sed que te hace crecer. El artista que aspira a una gran fertilidad debe recurrir a lo fértil para aprender y a lo fertilizante para prender. 

%-----032-----}}}
%-----033-----{{{
\section*{033}

Habría que nombrar a la humanidad como Patrimonio de la Humanidad para protegerla de ella misma. 

%-----033-----}}}
%-----034-----{{{
\section*{034}

La Tierra debió cometer algún pecado muy grave, porque Dios le envió la plaga más mortífera jamás concebida.

%-----034-----}}}
%-----035-----{{{
\section*{035}

¿Qué deben pensar las estrellas cuando un hombre lamenta su soledad ante ellas, invocándolas con términos gregarios y estadísticos como constelación o galaxia? Precisamente ellas, las más solitarias y aisladas, ¡rebañizadas por un pequeño microbio colonial!

%-----035-----}}}
%-----036-----{{{
\section*{036}

Si por tu comprensión frenas tu ímpetu ante la decrepitud de un viejo debes comprender también que el freno de ese viejo es más poderoso que tu ímpetu. 

%-----036-----}}}
%-----037-----{{{
\section*{037}

Por una parte, es inteligente el que sabe resolver problemas. Por la otra, saber resolverlos es un problema en sí mismo. ¿Qué clase de inteligencia resuelve este último?

%-----037-----}}}
%-----038-----{{{
\section*{038}

El objetivo de todo verdugo vocacional es ejecutar lo más limpiamente posible. Su ideal es una cuchilla infinitamente afilada, que traspase la carne sin resistencia alguna, es decir, sin dañar una sola molécula de ella.

%-----038-----}}}
%-----039-----{{{
\section*{039}

No hay muertes que se superan y muertes que no, sino muertes que se quieren superar y muertes ante las que nos aterroriza olvidar su terror. Superar la muerte es relativamente sencillo. Lo difícil es quedar por debajo de ella.

%-----039-----}}}
%-----040-----{{{
\section*{040}

Para algunos la magia es sacar una bola de una oreja, desaparecer tras un velo o salir volando en una escoba. Yo prefiero verla como cera que se saca de un oído, como la aparición tras el velo o el pisar de los pies en la tierra. Magia no como el oficio de hacer hechizos, sino como el ser capaz de vivir hechizado, y sobre todo magia como el arte de evitar (o hacer desaparecer) a los desencantados y sus desencantamientos.

%-----040-----}}}
%-----041-----{{{
\section*{041}

La concentración no es un templo, sino el jardín de fieras que lo rodea, lo prepara y lo protege. Una vez dentro del templo hay que olvidar al jardín, pues ya estamos protegidos y concentrados. Nuestra mente ya está limpia y preparada. Si en algún momento salimos de nuevo al jardín sin permiso las fieras nos atacarán, pero dentro estamos a salvo. ¿Y ahora? Mil veces hemos escuchado las técnicas  que conducen hasta allí, prometiendo grandes virtudes para el que consiga entrar y permanecer, pero ¿quién nos describe las estancias y los pasillos internos de dicho santuario? Estamos ya entonando nuestro mantra lleno de emes, esperando conocer las visicitudes de un lugar tan sagrado, pero empezamos poco a poco a sospechar que algo no funciona, que nos quedamos en una antesala en la que no hay accesos al interior. O peor, que dentro del templo no hay nada. Entonces la concentración se desenmascara y se muestra como fin, para decepción de los que la buscaron como un medio. Las almas concentradas se empiezan a llenar de vacío y estas se regocijan en su obra, que no es otra que la de mantener su templo limpio de ruido y suciedad. Limpio de todo. Pero hay una de esas almas que sale de allí con la sensación de haber sido estafada. Las fieras se le tiran encima pero descubre que estas no son tales. Vuelve el ruido y la mugre, pero el alma fugitiva ya no piensa en limpiarse, y menos aún en concentrarse. Decide llenarse con contenido, con creaciones que encuentran en el azar del ruido y la podredumbre una fuente de inspiración. A veces duda y se detiene, y todo a su alrededor lo aturde, pero descubre que mientras hace, crea y construye, su entorno es el del templo que siempre imaginó. Entiende entonces que el arte de concentrarse solo se perseguía a sí mismo como forma de purificación, mientras que el artista que se ensucia encuentra en el barro la materia prima de su obra y en la acción la única llave que necesitaba. 

%-----041-----}}}
%-----042-----{{{
\section*{042}

Demasiado tiempo y ninguno lleva el mundo pasando el rodillo de la apisonadora lineal por todas partes. La unidad asfaltada nos aboca al infinito y nos acribilla a discontinuidades. El salto mágico del 2\(\pi\) al 0. El de t=T a t=0. El de \(0.\overline{9}\) al \(1.\overline{0}\). Todas las matemáticas llenas de roturas y descosidos. ¿Y la física? Víctima orgullosa de ello. Pero tarde o temprano, que es lo mismo, habría que darse cuenta de que la unidad debe ser cíclica, de que debe tener una ida y una vuelta y que cuando el ciclo se complete nada en realidad debe haber sucedido. Del 0 hay que volver a llegar al 0 y recuperar la suavidad perdida. No es una tarea fácil, pues llevamos demasiado tiempo y demasiado poco encadenando años y siglos como orugas procesionarias. Ni el número ni el espacio ni el tiempo deben seguir entre costuras. La unidad debe ya acabar su fase larvaria, eclosionar y salir volando hacia otros lugares.

%-----042-----}}}
%-----043-----{{{
\section*{043}

A veces el cuerpecito aplastado de la hormiga, la rotura fatal en la casa del caracol o el esfuerzo vano de la mosca en la tela de araña me estremece hasta niveles abismales. Su dolor lo siento como mío de una forma total, con una conexión absoluta. Cada pequeña muerte se me hace insoportable y mi corazón quiere romperse con el de los demás cada vez que ve a unos ojos afrontando la angustia vital. Sin embargo, a veces deseo que todo el mundo se acabe, que reviente en añicos y que toda la amargura que hay en él se esparza y diluya por la galaxia. Quiero que todo muera y que no haya más lugar para ojos que miren con angustia. Que todos al fin veamos esa angustia y que no quede nadie para verla en otras miradas. Que todo se extinga de forma radical. A veces daría mi vida a cambio del latido más microscópico. Otras acabaría con toda la vida en la Tierra. Y nunca he visto contradicción alguna en ello.

%-----043-----}}}
%-----044-----{{{
\section*{044}

Busco la manera de contagiarte mi pasión a través de unas palabras que quiero que sean espermatozoides literarios. Busco que leyendo esto ardas de fiebre como yo, que sufras el nudo de la alegría en la garganta. Que quieras llorar y reír a la vez y descubras que ambas cosas son la misma. Deseo hacerte bailar con mis frases al son de mis latidos. Deseo llevarte allí donde tus defensas se rindan, hacerte trepar por el hilo fino de una pasión abstracta que quiere concretarse y desparramarse. Quiero hacerte sentir lo que yo siento y para ello te escribo miles de letras llenas de fecundidad isótropa que ya no caben en mi pecho. Justo ahora algunas de ellas llegan hacia ti como diminutas partículas de polen y se instalan en tu cuerpo queriendo hacerte reaccionar, buscando fecundar y multiplicar tus pasiones. Túmbate y léeme lentamente. Recorre cada una de mis frases y focaliza tus sentidos en las cosquillas que te hacen en la piel de los oídos. Déjate violar por mis ideas y asume que parte de mí ya vive en tus entrañas. 

%-----044-----}}}
%-----045-----{{{
\section*{045}

La inteligencia artificial es todavía un juego de inteligencia y un fuego de artificio, pero un juego y un fuego al fin y al cabo...

%-----045-----}}}
%-----046-----{{{
\section*{046}

Es muy difícil encontrar una sola canción ligera que no tenga una letra contaminada de dogmatismo cargante. ¿De qué sirven unos arreglos impecables si estos acaban aplastados por palabras pueriles sin arreglo posible? Un verso cayendo en dogma equivale a un swing cayendo en la primera corchea. ¿Cuándo llegará el swing a lo conceptual? La ingravidez en las palabras también es necesaria si se aspira a un género realmente ligero.

%-----046-----}}}
%-----047-----{{{
\section*{047}

¿Puede forzarse la filosofía? ¿Puede uno levantarse por la mañana y decir que se va a poner a filosofar? ¿Es posible ganarse un pan filosofal de cada día que no esté petrificado, u ordeñarse las ubres mentales periódicamente sin quedarse seco? ¿No es este arte más veneno que pan, más de serpientes de digestiones lentas que de rebaños lácteos? Pensar es una depredación de lo aleatorio, sin demonios de Maxwell. Cuando uno se encuentra con un filósofo productivo le asaltan las sospechas, y estas normalmente se confirman en forma de moral, la cual permite extraer un oficio a costa de perder lo esencial, la libertad de lo aleatorio. Un verdadero filósofo que un día quiera exprimirse un poco más solo tiene una opción, que es subir su temperatura un poco para, ganando algo de locura, obteniendo también más oportunidades. Pero la filosofía perfecta es eso: difusión blanca y pura, realizar excursiones en todas las direcciones, hasta el infinito, sin llegar nunca a nada.

%-----047-----}}}
%-----048-----{{{
\section*{048}

¿Hay una moral en la lógica? ¿Es lo contradictorio malo y lo coherente bueno? ¿De dónde viene semejante sinsentido?  ¿Acaso la contradicción no es un fenómeno natural? ¿Por qué, entonces, la lógica, siendo un
intento de dar sentido a la naturaleza, no la incluye en su seno? Pongamos que \(A=B\), \(B=C\) y \(A\neq C\). ¿Es esto una contradicción intrínseca, absoluta, o simplemente necesita una lógica más elevada para ser dada por buena? Pero si existiera un sistema semejante, ¿no sería como tener una moral en la que ningún acto pudiera ser verdaderamente malo? Estas son mis bases para una lógica lógica y una ética ética.

%-----048-----}}}
%-----049-----{{{
\section*{049}

La coherencia en la política es inmovilizadora. Si para llegar a un objetivo ideal hay que esperar a cambiarlo todo a la vez para que no coexistan leyes contradictorias entre sí, el salto que se necesita es entonces demasiado grande. Esto lo saben muy bien los que no quieren cambiar nada y se aprovechan de ello. Apelarán siempre a la falta de coherencia como elemento paralizador y por tanto debilitante. Admitámoslo, es mucho más fácil avanzar consiguiendo resultados parciales, por mucho que les duela a los puristas. Sin embargo, hay un precio a pagar si esta filosofía se lleva demasiado lejos: pasos más pequeños permiten una mayor reversibilidad, y en el límite del obstáculo totalmente asequible nos damos de bruces con la deriva aleatoria. Hay, por lo tanto, un régimen óptimo para acercarse a lo ideal de forma práctica.

%-----049-----}}}
%-----050-----{{{
\section*{050}

Al tener "un número" tenemos en realidad dos. En primer lugar tenemos "un" número. En segundo tenemos el "número" en sí. ¿Cuál de los dos es más fundamental? En "un número" tenemos "un" número como cantidad y un "número" como unidad {\it física}  . Por mucho que se quiera idealizar la unidad matemática, por mucho que quiera trascender a la propia naturaleza, una unidad lleva siempre implícita una magnitud, por oculta y fantasmal que parezca. El "número" es la unidad física de "un" número, lo mismo que un "metro" es la unidad de "un" metro. En el origen de todo esto está la cosa. La cosa permite que haya número, porque puede haber varias y porque puede ser subdividida en subcosas. Las cosas discretas, al fin y al cabo, dan luz a los números naturales. Pero ¿qué es una cosa? Un contraste abrupto y aproximado en el dominio de lo real y lo continuo. El contraste en lo continuo funda la cosa discreta, y esta al número natural, de la misma forma que lo continuo está fundado por infinitas cosas discretas. Precisamente el número nace como operación para anular el contraste y evaluar {\it cuánto}   cuesta dicha anulación. Por ejemplo, un 3 y un 2 sirven para que tres manzanas pesen igual que dos granadas, es decir para establecer dos escenarios idénticos (en cuanto al peso). El número es la acción que hace las cosas indistinguibles y la medida de lo que hay que hacer para conseguirlo. Al final, solo hay un contraste físico que estimula una serie de acciones para encontrar la perspectiva desde la cual no pueda apreciarse. El número como una búsqueda de lo igual, como la relajación de una tensión sensorial.

%-----050-----}}}
%-----051-----{{{
\section*{051}

Un profesor tiene mucho que aprender de un stripper. La profesión de ambos, después de todo, es {\it enseñar}  . Pero el hecho en sí de este verbo es enormemente trivial en ambas disciplinas. Lo difícil es, una vez que hay un buen contenido preparado, saber desnudarlo. Lentamente. Con gracia y cierto orden. Es este proceso de danza progresiva lo que destrivializa la explicitud de la materia escondida. Es la lencería la que, cubriendo y ornamentando, hace desear al contenido. Enseñar, estrictamente, se reduce a un instante fugaz tras el cual, si el estudiante arde en deseos de amar el cuerpo de la doctrina más allá del desnudo, debe empezar a recorrer a solas el camino que conduce a las entrañas. Justo ahí acaba la tarea del profesor y empieza la del amante. El primero debe limitarse a invitar, a seducir. El segundo, si se siente atraído, ya se encargará de comprender.


%-----051-----}}}
%-----052-----{{{
\section*{052}

De la misma forma que hay fibras musculares rápidas y lentas también se diría que hay neuronas que parecen más veloces que otras. Y tal y como sucede en el músculo, parece que puede haber conversión entre los dos tipos. Una mente de velocista puede acabar siendo más lenta que un cordero. Y viceversa. Hoy abunda el consumo de carne roja y tampoco escasea la mente del mismo color. Al igual que la deriva galáctica, casi todo parece correrse hacia ese color. Se nota hasta en la obra de los artistas. Producen con periodicidad y oficio. Sin asperezas propias de una pluma explosiva. Algunos pensadores casi nos están llevando al límite de la filosofía aeróbica, ejercicio de la mente isométrica. Así, por supuesto, queda al alcance de cualquiera, porque no salta ni vuela a ningún lugar. Tampoco produce lesiones violentas, aunque sí desgastes por erosión. El pensamiento cae en la rutina de tareas pesadas y repetitivas hasta convertirse en una superficial tarea de fondo. Filosofía al ralentí. Pensadores horizontales capaces de permanecer mucho tiempo sentados, olvidando la verticalidad que requiere la altura y la profundidad. Contra las ultramaratones de la dialéctica, el aforismo como estela de una mente a explosión.

%-----052-----}}}
%-----053-----{{{
\section*{053}

Los libros de filosofía casi siempre decepcionan porque lo que uno suele buscar en ellos no es solo la palabra de una mente fuerte, sino la vía para fortalecerse uno gracias a esa palabra. Los libros de autoayuda en cambio, también decepcionan, porque uno suele encontrar en ellos vías de fortalecimiento diseñadas por una mente débil. 


%-----053-----}}}
%-----054-----{{{
\section*{054}

En la divulgación científica el autor se suele negar a usar ecuaciones escudándose en una estúpida cita atribuida a Hawking, pero la verdadera razón para no hacerlo es otra y además doble. Por un lado menosprecian a su público pensando que no será capaz de entenderlas. Por el otro, demuestran su pereza e incapacidad de explicarlas de una forma que cualquiera con paciencia e ilusión pueda alcanzar cierta comprensión sobre ellas. Un lector que quisiera fortalecerse con el conocimiento de la naturaleza acaba mal informado, intimidado y menospreciado, víctima de la incapacidad, prepotencia y cobardía del escritor y las exigencias del editor. Las matemáticas, al ser el instrumento que permite describir la naturaleza, deberían ser protagonistas en cualquier libro de divulgación que se precie. Si con ello se pierde algún lector no pasa nada. Sí pasa que los que quieren aprender y no se detienen ante el primer obstáculo acaben aburridos por unos criterios que benefician al lector vago y mediocre. En definitiva, la divulgación de calidad debería ser en realidad material científico, mientras que la ciencia, si la escribieran personas verdaderamente competentes, siempre debería tener ya una calidad pedagógica. Divulgar significa hacer autopistas, no autopsias, y así ampliar el camino que el investigador ha recorrido de forma tortuosa hacia algún lugar sorprendente, y no simplemente enviar una bonita postal desde allí. Divulgar no debe ser enseñar con vulgaridad unos caminos nada vulgares, sino dirigirse al vulgo para que este lo sea menos. La divulgación actual te deja, como su nombre indica, doblemente vulgar.  


%-----054-----}}}
%-----055-----{{{
\section*{055}

Un atleta corre en su vida dos carreras. Una es la carrera contra el tiempo. La otra contra sus adversarios. De esta última hay muchas. De aquella solo una. En la segunda se labra un palmarés. En la primera no hay vitrinas que valgan. Cierto es que solo los buenos de verdad perciben esta dualidad, pues a los atletas mediocres solo les queda el consuelo de ganar a otros atletas y nunca al propio tiempo. En la carrera científica pasa lo mismo. Hay dos carreras. Solo se habla de una porque los que hablan son siempre los del segundo grupo. ¿Para qué darse una cita con la Historia cuando uno puede dedicarse a cosechar un historial de citas? En atletismo se admite que las grandes marcas solo están al alcance de los jóvenes. Cuando ya no son competitivos se retiran. Algunos se quedan de entrenadores o comentaristas, pero pasan a segundo plano sin esconderlo. En cambio en ciencia casi nadie se retira. Algunos se jubilan por ancianidad, pero nada más. Lejos de eso, cuando sus fuerzas flaquean les concedemos cátedras para que ya no se levanten. ¿Y qué clase de carrera se corre sentado? Desde allí se dedican a poner vallas a las competiciones de promesas mientras ellos acaparan todos los recursos. A los jóvenes se les seduce a malvivir con la promesa de que cuando se vuelvan decrépitos podrán también reinar desde una de esas sillas seniles. El objetivo de la carrera acaba siendo acabar para poder sentarse. ¿Dónde queda la primera carrera, la que trepa sobre hombros y cabezas de enanos y gigantes? ¿Es la ciencia directamente una disciplina paralímpica? Menos mal que antes de hacerme científico fui atleta y, aunque mediocre, aprendí cuál es mi verdadera carrera. 

%-----055-----}}}
%-----056-----{{{
\section*{056}

Durante mucho tiempo se ha pensado que el camino de la lucidez no tiene salida, que al contrario que otros túneles paranormales, este es luminoso con una oscuridad al fondo. El lúcido descubre, o eso afirma, la inutilidad de la vida, su falta de sentido, y cae en el pozo nihilista del que ya no puede salir. Carente de fe, acaba débil y atrofiado, lamentándose del resultado penoso que le ha regalado su lucidez. Pero ¿es este el final del trayecto al que lleva la razón más estricta? ¿Es esta oscuridad el destino determinista de aquel del que más luz es capaz? ¿No será que dichos lúcidos se han extraviado y su orgullo no les permite reconocerlo? ¿No habrán pasado por alto un error en su riguroso razonamiento que les haya llevado hasta consecuencias demasiado bajas? ¿No serán estos pobres lúcidos familiares lejanos de William Shanks, emulándolo no ya en el paso 528 sino en el primero? Cuando afirman que la vida no tiene sentido ya están avanzando en una concatenación condenada al error que les acaba llevando a una vida tan penosa que creen que con ella, con su propia falta de sentido, demuestran de forma empírica su tesis primordial. Ellos presumen de que su lema es la consecuencia de observar y razonar. Yo creo más bien que dicho lema constituye su premisa y también su merecida consecuencia. Parten de un error y sus vidas acaban partidas. Mejor harían en estudiar primero qué es la vida antes de prejuzgarla de forma tan precipitada. Y si no sabemos en qué consiste, ¿cómo atrevernos a tacharla de sinsentido? ¿Cómo nos atrevemos a ensuciarla con nuestros prejuicios de moralidad lineal? ¿Por qué, de haber un sentido, tiene que ser lineal? La evidencia abrumadora es que el sentido de la vida no es lineal, sino cíclico. Por tanto, la fe vital y la fuerza que de ella debemos obtener también deben ser recurrentes. Dejémonos de patrañas cristianas de una vez por todas. El lúcido, el escéptico, ha fallado al no desconfiar de lo lineal, y por eso no ha podido hasta ahora convertirse en una criatura llena de fortaleza. ¿Cómo podría la verdadera lucidez permitir otra cosa que no fuera la fuerza, si son precisamente la fuerza y sus ciclos los responsables de todo destello? Eso sí, no es fácil albergar una fe cíclica en el corazón. No valen trampas helicoidales ni espirales suicidas. Hay que convivir con una fe desesperada, descorazonadora, escéptica si se quiere, pero nunca impostora ni debilitante. Dicha fe no promete cumplirse una vez en el ocaso sino que se cumple incesantemente. Ni siquiera requiere que seas un iluso ciego, sino alguien que se mantenga fiel a su centro de gravedad con los ojos bien abiertos. Esto es lo que la lucidez y la escuela del escepticismo tenían que haber descubierto ya. La devoción de la vida hacia sí misma, la religión de la fuerza que usa la razón para no perderla. 


%-----056-----}}}
%-----057-----{{{
\section*{057}

Dicen que los borrachos nunca mienten, pero ¿quién lo dice? Borrachos también. Lo que sé de todos ellos, a ciencia cierta, es que nunca, jamás, {\it escuchan}  . Quizás por eso creen que solo dicen verdades. Las aguas turbias tienden a creerse profundas porque apenas pueden verse a través de sí mismas, y cualquier profundidad se disfraza de fondo. La nitidez del agua cristalina no crea leyendas ni puede engañarse con falsas profundidades. Pero no nos engañemo: el camino de la claridad no es ni menos embriagador ni más salubre que otros. 

%-----057-----}}}
%-----058-----{{{
\section*{058}

En toda persona ambiciosa hay dos caminos a recorrer. El primero es el común, el de la edad cronológica. El segundo, más tortuoso y diferenciador, es el de tu obra. Ambos tienen un clímax natural y es imprescindible hacerlos coincidir. 

%-----058-----}}}
%-----059-----{{{
\section*{059}

No sabemos definir el momento exacto de la muerte. Para cuando esta ya es segura muchos órganos y tejidos ya no sirven, así que hay que cortar antes, {\it por lo sano}  , para poder sanar a otros. Para que tu corazón siga latiendo en otro cuerpo no se puede esperar ni a la guadaña ni al determinismo. Hay que ser práctico y despiezar {\it  in vivo}  , como en cualquier otra depredación. Esto, lejos de ser triste, debería arrancar una sonrisa incluso al que le acaban de arrancar sus órganos y aún le quedan unos instantes para esbozarla. 

%-----059-----}}}
%-----060-----{{{
\section*{060}

Especializarse quiere decir profundizar en un tema de forma cada vez más superficial, y hacerse superficial de forma cada vez más profunda. 

%-----060-----}}}
%-----061-----{{{
\section*{061}

Cuando se te muere tu hijo la sociedad te dirige su mayor piedad. Muchos te dirán que es normal si nunca lo superas. Te comprenderán tanto, serán tan conscientes de tu infinito dolor que si un día te sientes un poco mejor, olvidas un rato o incluso sonríes, te sentirás mal al pensar que no es eso lo que se espera de ti. Tu pérdida es muy grande, pero los demás te la querrán multiplicar. De tanta comprensión acabarán comprimiéndote. Y puede que fueras alguien que apenas veía a su hijo. Quizás hasta te llevabas mal con él. Puedes que sintieras que no te quería apenas. O incluso puede que fuera tan pequeño al nacer, si es que había llegado a hacerlo, que casi no habíais compartido nada todavía. A los demás les dará igual. Seguirán con su mantra de que no hay pérdida peor y te criticarán por la espalda si das signos de mejoría o retiras tu luto. Cuando se te muere tu perro la cosa es diferente. Si con el hijo te ves obligado a llorar, con el perro te dará vergüenza hacerlo. Alguien te comprenderá parcialmente pero la mayoría no considerarán que hayas perdido nada. Es más, si te ven triste más de un día muchos se reirán de ti. Puede que ni siquiera tus amigos, tu familia o tu pareja te acompañen en ese dolor. Nadie te va a conceder un día sin trabajo por perder a tu mascota. Tendrás que esconder y disimular tu dolor. Si con tu hijo profanaban tu pena con tu perro la sentirás en soledad. Y quizás ese perro te dio toda su vida durante más de quince años. Puede que te regalara toda su atención, que cada día te recibiera innumerables veces con grandes fiestas. Es posible incluso que pasearas infinidad de horas con él por medio mundo. No es descartable que tu hijo te amargara la vida y que tu perro te la alegrara. Tampoco es una locura pensar que el perro te quería mil veces más y mejor que tu hijo. No tiene por qué ser el caso, pero no es un escenario imposible. Y si así fuera, ¿quién se atreve a decir en voz alta que quiere más a aquel con quien comparte menos genes? ¿Confesarías que tu segunda pérdida supone más que la primera? Si fuera tu caso y te atrevieras a decirlo, ¿quién tendría derecho a criticarte? ¿Por qué deben los demás juzgar las jerarquías de tu corazón? Ante toda esa escoria puedes fingir lo que quieras, pero a ti mismo no te engañes. No finjas ante el espejo ni ante las fotos. Tampoco ante tus recuerdos. Si lo haces ya se encargarán tus sueños de decirte la verdad.


%-----061-----}}}
%-----062-----{{{
\section*{062}

Esta mañana tuve una idea pero la he perdido. Sé que no se me ha caído al suelo y que seguramente no se ha roto. Simplemente no la encuentro. Si fuera una aguja en un pajar podría recorrerlo sistemáticamente y sé que con paciencia podría encontrarla. Pero para mi idea no tengo forma de rastreo. Puedo esforzarme, pero ¿hacia dónde dirijo mi esfuerzo? Intento evocar recuerdos de esta mañana para acercarme al lugar donde se me perdió pero no hay forma. Y sé que está ahí, incluso creo que de aquí a muchos años aún seguirá ahí. Lo sé porque si la veo la reconoceré. Pero sigue aislada y perdida dentro de mí. Decido entonces sustituir la idea por la idea de hablar sobre la pérdida y búsqueda de la idea con la esperanza subrepticia de que la idea primera vuelva a mí muerta de celos. Llego incluso a pensar, conformista de mí, que el haberla perdido me ha podido originar una idea mejor que la que perdí. Pero todavía conservo el gusto, el sabor agradable que me dejó en el paladar antes de que se desvaneciera su rastro. Huelo su aroma cada vez más tenue sin poder reconstruir el camino de vuelta. Muchas veces ese aroma me sirve para recordar que he tenido un buen o un mal pensamiento. Este regusto dura más que el propio pensamiento. Pero no hay nada que hacer. La he perdido, o mejor dicho, sé dónde está pero se ha borrado el camino que lleva a ella.

%-----062-----}}}
%-----063-----{{{
\section*{063}

Dios es un atractor extraño: orden y caos, divergencia y periodicidad, todo en un mismo concepto.

%-----063-----}}}
%-----064-----{{{
\section*{064}

Despide a la {\it acción pasiva}   de tu vida. Haz tus planes. Organízate de la forma más eficiente posible. Sea tu disciplina una coraza. Piensa que ni un solo minuto debe ser desperdiciado. Usa hasta el último de ellos para crecer, conocer, hacerte fuerte y crear todas las obras que quieran salir de tu talento. Planifica tu vida de forma que no des ni una sola concesión a las interrupciones. Ni se te ocurra ceder una sola hora fuera de tu camino. Desprecia la mano que te tiende el viejo y el guante que te ofrece el joven. Tú sigue tu senda con mano férrea y pie impertérrito. Ten prisa por hacerte grande y fuerte y olvídate del mundo. Pero... ¿para qué tanto prepararse? ¿Para qué tanta estructura? ¿Por qué atesorar tanta energía en la oscuridad de tu pecho si tienes que rechazar todas las manos que llamen a tu puerta? ¿A santo de qué elevar tu belleza día tras día, subir montañas y construir cumbres que nadie pueda apenas vislumbrar? Si sigues ese camino, tu corazón será un día una joya preparada para contestar a la única llamada a la que no hay que ignorar. Una llamada que no habrías sabido apreciar de haber tenido siempre la puerta abierta. Tu corazón cerrado estará lleno y listo para abrirle la puerta a la única voz que debe interrumpir tu ascenso. Toc-toc. ¿La oyes? ¡Ya está ahí! Todos estos años de disciplina están a punto de dar su fruto. ¡Corre y descorre las cortinas, sube las persianas de par en par! Quien siempre te ha seguido de cerca ha llamado a tu puerta y debes entonces lanzarte a ella desde tu gran altura y dejar que emanen todas las maravillas que has atesorado. Ella verá que eres merecedor de sus riquezas y te llevará consigo. Será el momento de partir. Toda tu disciplina y tu grandeza habrá encontrado por fin su cometido, que no es otro que el de abandonarse por completo a la {\it pasión activa}  .

%-----064-----}}}
%-----065-----{{{
\section*{065}

Llevar a cabo grandes obras es tarea de genios. En cambio, tener proyectadas, incluso empezadas, no ya unas pocas sino una infinidad de grandes obras es algo muy común. Estos pseudo genios creen tanto en el valor de sus ideas que acabarlas se vuelve secundario, pues ellos ya se consideran genios sin necesidad de constatación externa. Pero al final solo queda lo acabado, o casi. El resto de grandes ideas se vuelven a su limbo con la misma volatilidad que un sueño. ¿No será porque las soñamos precisamente el motivo por el cual pensamos que son geniales? Si eres de los que tienen grandes cosas entre manos, o eso crees, date un pellizco. Tus cábalas... ¡acábalas!

%-----065-----}}}
%-----066-----{{{
\section*{066}

Es muy difícil hablar con propiedad de Nietzsche. Su obra y vida es algo mucho más complejo que cualquier cálculo de relatividad general. Sin embargo los profesores de instituto no hablan alegremente del tensor de Ricci. Tampoco ves en el cine películas sobre geometría diferencial haciéndola parecer euclidiana. ¿No sería ridículo? Pues eso es lo que hacen con Nietzsche: hablar mucho de él con un gran desconocimiento, el cual puede seguir ahí a pesar de haber leído su obra completa diez veces. Tras su célebre y poblado bigote se esconde la mente más brillante que nunca ha habido. A su lado Einstein es puro parvulario y Marx un ser unicelular. Pero eso no quiere decir que haya que dejar a Nietzsche en manos de los que se doctoren en él. El filósofo de Leipzig no sirve para la academia, y menos aún para los colegios de secundaria. Si quieres conocerlo debes hacer caso únicamente de lo que él mismo te diga en sus propios libros, y aun así tiene la gran habilidad de engañarte, porque en vez de usar tensores usa palabras, y en vez de utilizar un lenguaje arcaico para distanciarse te seduce con una prosa fácil y directa. Eso provoca que se le acerque todo el mundo, que toda la chusma se atreva a hablar de él e interpretarlo. No tenía a su disposición el pathos de la ecuación abstracta y consciente de ello desarrolló el pathos del falso entendimiento. Mientras el universo se dejaba explicar por relaciones geométricas, el espíritu, universo tan vasto, lejano y misterioso como el otro, o más, siempre se ha resistido a ello. Nietzsche se aproximó a él de una forma muy especial: dirigiéndose a ti directamente, sin intermediarios profesionales ni académicos marmóreos. Nietzsche te pide siempre una cita a solas con él. Te lo pone fácil de entrada. Pero cuidado, porque sin el artificio del tecnicismo tratará de reducir tu sentido común a añicos. Para que lo recompongas haciendo otro mejor. Repito, casi nadie puede hablar con propiedad de Nietzsche, porque Nietzsche es, en principio, propiedad de todos. Así lo quiso él, para poder percurtir en todas las almas y seleccionar aquellas que resonaran con mejor timbre, demoliendo a las que se oyeran huecas. La vida y la obra de Nietzsche es tan compleja porque en ellas no basta el entendimiento. Da igual cuánto gastes tus codos en sus libros. Lo que importa en ella es hasta dónde eres capaz de abrir tus entrañas para dejártelas remover, examinar, ridiculizar, reorganizar...


%-----066-----}}}
%-----067-----{{{
\section*{067}

Los vampiros solo toman un poco de sangre. Beben lo que necesitan y dejan con vida a sus víctimas. En cambio, los no vampiros torturamos a las nuestras, las matamos y luego buena parte de ellas la tiramos a la basura. Consideramos las películas de vampiros un género de terror. Muy bien, ¿a qué clase de género pertenecemos nosotros? El príncipe Vlad, cuando era humano, empalaba a sus víctimas. Luego, como no-muerto, se hizo mucho más sostenible. ¿Será ese su secreto para vivir siglos? ¿Alguien ha visto a algún príncipe de las tinieblas obeso o diabético? ¿Se considera vegano a un vampiro?

%-----067-----}}}
%-----068-----{{{
\section*{068}

Es tan típico ver a adolescentes con un ejemplar de ``El Anticristo'' bajo el brazo, acompañando a otros objetos y ropajes supuestamente oscuros y tenebrosos. Qué poco saben lo que es la oscuridad y las tinieblas... Una vez más, Nietzsche supo hacer marketing y conseguir que muchos leyeran tras un título supuestamente satánico una de las obras más profundamente cristianas que existen. ¿Y acaso no es mil veces más gótico y espeluznante Jesucristo y su herencia que el propio Satán, el cual no deja de ser una caricatura de cartón piedra? El autor, tras el título, ya avisa de que el texto es para una selecta minoría, pero el adolescente siempre se creerá dentro de ella (condición necesaria pero no suficiente para estarlo realmente). Más tarde el lector se desespera porque el libro ``solo'' habla de ética y moral... Algunos lo abandonan. Otros no, para poder decir que han leído ``El Anticriiiiistoooooo'', confiando en que su interlocutor desconozca esa obra. De esta forma el autor tiene un gran público debido a un malentendido intencionado. No hay bestia satánica tresillos de seises, sino un hombre cristianísimo que analiza no la inversión de la cruz, sino la de todos los valores. 

%-----068-----}}}
%-----069-----{{{
\section*{069}

Es revelador que no haya, o que no sea popular, un verbo específico para decir ``no madrugar''. Como si lo natural fuera levantarse tarde. Cuando uno observa las criaturas diurnas parece que no se levantan con fastidio. Al revés, se muestran inmersas en una vitalidad máxima. Al decir que alguien madruga parece que hablamos de una anomalía. Todo por haber normalizado al trasnochado y al insomne. La excepción se apodera tanto de la norma que madrugar acaba pareciendo aberrante. De esta forma un defecto doble y encadenado acaba pareciendo lo natural. Levantarse cuando marca nuestra fisiología sana se ha convertido en algo digno de un premio divino. A quien madruga Dios le arruga, dicen algunos. Al final se impone el estilo vampírico sin vida eterna, la transición horizontal de la cama al ataúd sin apenas haber conocido la verticalidad solar.

%-----069-----}}}
%-----070-----{{{
\section*{070}

Repetirse en síntoma de senilidad y decrepitud. Volver siempre sobre lo mismo es, sin embargo, el signo más inequívoco de lucidez y plenitud. 

%-----070-----}}}
%-----071-----{{{
\section*{071}

A menudo veo en la carretera cómo transportan a centenares de criaturas en un camión de deportación, hacinados en su camino
al matadero, recibiendo quizás entre rejas y maderas su único rayo de sol, o su único olor placentero de alguna flor remota.
 Nosotros vemos nuestra vida como un largo viaje entre dos procesos puntuales: el nacimiento y la muerte.
Para estos animales en los que nacimiento y muerte están tan explotados e hipertrofiados es totalmente al revés: su vida es un viaje puntual
entre dos largos y agónicos procesos. 

%-----071-----}}}
%-----072-----{{{
\section*{072}

Leer es un verbo demasiado correcto. Estrictamente hablando consiste en recorrer
todas las líneas de un texto. Ya el diccionario demuestra lo aséptico del término al definirlo como ``pasar la vista
por lo escrito''. Un proceso frío y neutral, un mero paseo de la vista, sea esta gorda o no. ¿Se puede entender la 
lectura de otra forma, de una manera menos higiénica? El diccionario puede irse a paseo: leer es vivir la vida de lo vivido,
y también la de lo nunca vivido pero vivamente imaginado. Leer es un proceso sucio y subversivo, que además enseña
a ver la suciedad y subversión del mundo en que se escribió lo leído. La lectura entra por la vista, pero penetra en todas las 
entrañas y en ellas cada cual cultiva su propia flora intelectual, su colección de microbios literarios, tan definitoria
del lector como su análoga fisiológica. ¿Un paseo visual? ¡Nunca! Leer está mucho más cerca de ser un tránsito visceral.

%-----072-----}}}
%-----073-----{{{
\section*{073}

Aún no sabemos si una teoría final de la física fijará los valores de las constantes universales o no, pero 
a veces me pregunto si también podría llegar a fijar la {\it base}   en la que estos números se expresan. 
¿Es el universo estrictamente binario? ¿Existe una base en la que todo se expresa de forma más elegante? 
Todo dependerá de cuántos dedos tiene Dios en las manos.

%-----073-----}}}
%-----074-----{{{
\section*{074}

Se pueden decir dos tipos de verdad: la científica y la sincera. A una se llega a través del ciclo constante
entre experimento
y teoría, pero ¿y a la otra? Todavía dependemos de que alguien quiera ofrecérnosla, aunque jure con la mano
en la biblia, y por
lo tanto la pena de muerte siempre puede matar a un inocente. Pero ¿y cuando podamos leer la mente algún
día, sin ningún margen de error? ¿Desaparecerá la justicia? Al que ha cometido una
atrocidad podremos conectarle un lector y descubriremos si realmente la ha hecho él. Lo que pasa es que
entonces podremos leer también {\it por qué}   la ha hecho, qué le ha llevado a hacer algo así. Empezaremos
a entender la cadena que conecta los términos ``persona'' y ``criminal''. Y la pena de muerte será algo todavía
más terrorífico, porque al tener acceso a la segunda verdad descubriremos que esta se diluye tanto entre
raíces tan profundas que no sabremos a quién castigar. Cuando eso llegue, la justicia no es que vaya a 
desaparecer, sino que por fin será justa. El día en que pueda verlo todo será el día en que podrá de verdad
jactarse de ser ciega.

%-----074-----}}}
%-----075-----{{{
\section*{075}

El final de ``Así habló Zaratustra`` es hermoso, pero un poco iluso, porque los llamados hombres superiores
nunca habrían huido despavoridos. El hombre superior ha llegado a un nivel de potencia extremo, a pesar
de seguir siendo un mezquino. Por eso nunca huiría al ver al león. En cualquier caso sería el león el 
que nunca se acercaría a ellos. De lo contrario, moriría abatido rápidamente. Cierto es que el hombre 
superior, capaz (y en ello está) de destruir la Tierra, no es nada si lo desnudas, pero el hombre ya
nunca va desnudo. Las primeras armas le dieron potencia para superar (en potencia) al resto de los animales. 
Luego las armas nucleares le han dado tanta que nadie ya puede ser superior en este aspecto, salvo
fuerzas físicas a escala geológica o astronómica. Por eso es iluso pensar que un león puede espantar a 
esa tribu de supuestos hombres superiores. Quizás Zaratustra toma el león como un símbolo de gran fuerza,
aunque ignoro qué clase de símbolo podría realmente echar a los visitantes del famoso y desconocido ermitaño. 
Podemos imaginar que el león coge desprevenidos a los hombres. De acuerdo, pero poco tardarían en desenfundar
sus armas y acabar con él. Mi pregunta es, ¿a qué clase de hombres espera Zaratustra entonces? ¿Cómo es posible
superar al que ya somete todas las fuerzas de la naturaleza a su necio antojo? Probablemente Zaratustra, como
yo, esté esperando a los únicos que pueden darnos todavía esperanza: los robots. 

%-----075-----}}}
%-----076-----{{{
\section*{076}

Hay algo mucho peor que presenciar el descomedimiento de algún atrevido. Una persona indiscreta puede
pecar de desatino, locura o descaro y a muchos les parecerá desagradable. Sin embargo prefiero eso mil
veces a presenciar la prudencia {\it en acción}  . Normalmente una persona que pasa por prudente
pasa también inadvertida, y eso le lleva a cultivar pocos enemigos. Pero cuando tengo la oportunidad
de cazar el instante preciso en el que una prudencia es ejecutada, siento la peor de las
repulsiones. Cuando adivinas en su mirada que algo está callando o dejando de hacer. Te das cuenta
que hay algo dentro que queda amordazado, y que sin embargo se deja notar. Ese prudente se me antoja
entonces como alguien oscuro, poco sincero, turbio... No te avisará si te cuelga un moco de la nariz 
o si se te ha enganchado una hoja en el pelo. No te dirá si algo no le gusta de ti. No querrá herirte,
pero te dejará expuesto para que te hieran los demás si lo creen conveniente.
A veces la pasividad se convierte en acción,
en la acción {\it por defecto}  , y es en esos momentos donde la prudencia es necedad pura. Cuántas veces
me habrá pasado que le abro mis verdades a alguien, en un momento complicado, y recibo silencio, silencio
prudente pero silencio al fin y al cabo. Es un silencio bomba, que te deja sin oír más que un pitido
porque en realidad su sonido es tan fuerte que acaba dañando tu capacidad para escuchar. A veces,
en vez de respuestas, recibo formalismos lacónicos y mortecinos, tan refinadamente correctos y prudentes que
deberían estar castigados con la horca. El prudente puede pasar por virtuoso, incluso durante años,
mientras no suceden cosas importantes, pero cuando el tiempo se vuelve violento, la prudencia
pierde su erre. En esas situaciones el prudente se siente desbordado.
Su mente discreta no puede o o quiere captar
y seguir el ritmo de señales rápidas y turbulentas. Pero incluso en ese revuelo su silencio
rechina como una locomotora oxidada. Así que ten cuidado con los prudentes porque son tan discretos
que cuando te apoyes en ellos tu acción no encontrará reacción, y te caerás. Se volverán
mudos, incorpóreos y transparentes. Existe otra prudencia más virtuosa, por supuesto, la de no iniciar
una acción que no debe ser iniciada, pero aquí hablo la prudencia que se basa en silenciar toda reacción
que debe ser reaccionada. Conclusión: cultiva la prudencia y huye de los {\it pudentes}  .


%-----076-----}}}
%-----077-----{{{
\section*{077}

Hace un tiempo me avergonzaba de los jóvenes que abuchearon a Savater en su visita a la Universidad
de Barcelona, y todavía lo hago, pero últimamente este filósofo venido a menos se está ganando a pulso
el ser cuestionado. Según él, algunas personas confundimos ``la sangre de las personas con la de los 
animales''. Una confusión que tiene bastante sentido científico, pues en esa sangre es mucho más lo que nos une a ellos
que lo que nos diferencia. Pero para el autor de ``Tauroética'', hay una discontinuidad entre lo humano y lo meramente
animal. Curiosa religión para alguien que tanto ha criticado a la Iglesia. Por supuesto que hay muchos que
lo critican y luego se comen alegremente un bistec, y otros que lo defenestran sin saber siquiera escribir
su nombre correctamente... Pero yo es que me pasé la adolescencia leyendo casi todos sus libros y aprendiendo
en ellos las bases del pensamiento crítico. Savater ha sido una puerta pequeña y generosa que conduce 
a grandes mentes como Nietzsche, Cioran o Voltaire. También ha sido un faro de humor y lucidez en una España oscura
y rancia. Es cierto que él siempre ha sido aficionado a los toros, pero ¿por qué muestra semejante animadversión 
contra aquellos que, lejos de rebatir los principios de la ética, simplemente quieren extender su alcance? 
¿O es acaso que nuestra ética solo es sostenible a costa de no tenerla con los demás seres? De ser así, eso no figuraba
en ningún libro. Desde un punto de vista más formal, creo que el problema viene de agarrarse al antiguo, casi bíblico
principio de que el hombre es superior al resto de criaturas de la Tierra. Este principio, a medida que avanza
el conocimiento científico, se demuestra cada día más falso, lo cual no quiere decir que los científicos sean más
éticos que los filósofos taurinos. Al revés: se cometen más y peores en masacres en laboratorios que en las plazas de toros.
A lo que me refiero es a que el conocimiento, aunque haya sido extraído a veces de forma penosa, nos conduce a la
certeza de que el resto de los animales siente y piensa, como nosotros. En algunos aspectos de forma menos sofisticada y en otros
más. Durante demasiado tiempo se les ha considerado
máquinas que actúan por instinto, sin libertad, pero hoy sabemos que aunque tienen instintos y su concepto de libertad
es dudoso, las personas también tenemos aquellos y tampoco sabemos siquiera definir lo que es esta. Y hay algo aún peor que
el sufrimiento particular que infligimos en un individuo: el sufrimiento colectivo sobre todo el ecosistema, que tiene
su máxima expresión en la industria cárnica, de la cual la tauromaquia es simplemente una faceta más. Señor Savater,
usted nos enseñó a muchos que teníamos que hacer ``lo que queríamos'', donde lo que uno realmente quiere es lo que
de verdad le interesa a largo plazo y a amplio juicio, no simplemente lo que a uno le apetece en el momento. Pues bien,
la tauromaquia es una de muchas expresiones en las que las personas estamos haciendo lo que no queremos, pues las
acciones basadas en el principio de que el hombre tiene a su disposición la Tierra y sus criaturas están demostradamente
obsoletas y nos están conduciendo no ya al exterminio del ecosistema sino también al propio. La ética que usted me enseñó tiene
como consecuencia natural el defender a la Tierra de aquellos cuyo comportamiento nos conduce a un futuro negro. Y le guste 
o no, la tauromaquia se ha convertido un doble símbolo. Por un lado su vertiente clásica: la lucha del débil pero astuto
bípedo y racional contra el poder y la bravura de los cuernos taurinos y supuestamente irracionales. Pero este aspecto
ya está de lo más caducado. Hoy es el bípedo la bestia irracional, y el toro se ha convertido en el abanderado que sufre
la tortura innecesaria, que padece el sacrificio de los excesos que el mono no tan astuto perpetra continuamente. La plaza
de toros se ha convertido en un escaparate relativamente amable de lo que sucede en los mataderos. Y ese símbolo es el que va a perdurar,
el de una época en la que los hombres disfrutaban haciendo sufrir innecesariamente. Cada día sabemos con más precisión que
el toro y la vaca son seres asombrosamente racionales y con una riqueza emocional enorme. Y cada día el ser supuestamente racional se
revela como más atroz y destructivo. El mito ha cambiado, y todos los que seguís defendiendo a ese arte pensando en el 
imaginario antiguo, el que leemos en El Caballero de Olmedo, estáis ya haciendo el ridículo, el mismo que hacen los que defienden
al toro pero luego se comen a la vaca. Yo sé cuánto valora usted
la naturaleza salvaje, y cuánto admira el mundo de las fieras y los parajes exóticos que tanto placer dan en sus 
novelas preferidas. Pero la industria de los toros, las vacas, los cerdos y un largo etcétera son los que, para dar
espacio a sus cabezas de ganado, están acabando con todo eso. ¿Dónde queda la ética en sus artículos? ¿Dónde queda
su lucidez? Claramente hemos entrado en una época en la que la ética debe luchar porque las personas, que tenemos
tanto poder, decidan hacerse sostenibles, y eso requiere aprender a respetar y a amar a todos los seres vivos. Podríamos resumirlo en
que necesitamos una ética basada en aprender a controlarnos, a regularnos. Cuando
recuerdo sus libros pienso que había en ellos demasiadas personas y demasiada poca tierra en ellos. La filosofía está muy
bien, siempre y cuando no entre en contradicción con la filia por nuestro planeta. Tuve la suerte de convertirme en una persona
librepensadora gracias, en parte, a sus libros y artículos, y ahora me doy cuenta que uno de mis mejores maestros no quiere llevar los
principios de su pensamiento hacia sus consecuencias. Estimado profesor Savater, usted y yo sabemos que hay animalistas que dicen,
con buen corazón, cosas filosóficamente torpes, pero piénselo bien: de tanto despreciar a los animales han acabado ustedes despreciando
aquello que los define, el ánima. Y qué curioso, porque por otra parte, diríase que los toreros y aficionados creen que son ellos quienes la
poseen y no su víctima. Piénselo bien, rectificar es de sabios y usted siempre me pareció que lo era. 


%-----077-----}}}
%-----078-----{{{
\section*{078}

En contra de lo que pueda parecer, una perspectiva marciana del mundo es la más razonable, humana y
amigable con la Tierra que uno pueda tener. Primero porque es la más abierta, la perspectiva con más
perspectiva. Segundo porque es la menos antropocéntrica. Y no hace falta que marcianitos verdes vengan
a visitarnos para adoptarla. Este punto de vista, aunque abstracto y remoto, nos hace sentir mucho más
reales y cercanos no ya a todos los que compartimos DNI, sino también a los que compartimos DNA, que es
mucho más importante. Solo la perspectiva extraterrestre, como las agencias espaciales nos llevan
enseñando décadas, permite observar la Tierra de forma global. Desde tierra firme vemos demasiado cielo,
nos centramos demasiado nosotros y nuestros puntos de vistan divergen entre ellos de forma peligrosa. Con la
vista forjada al calor de otra civilización no solo queda patente nuestra pequeñez, la cual nos cura
de la soberbia, sino que también aprendemos que lo humano es universal, pues ¿acaso no consideraríamos 
humana una civilización externa? Y cuando aprendemos a expandir lo humano hacia afuera también lo hacemos
hacia dentro. Lo humano se expande a todo lo que vive y todo lo que vive es también humano. El grado de
civilización de un planeta puede medirse por lo lejos o cerca que está de aprender esa identidad. 


%-----078-----}}}
%-----079-----{{{
\section*{079}

Las banderas parecen ser todas diferentes pero en realidad son todas iguales: la misma forma, el mismo 
mástil, la misma tela, el mismo tamaño, el mismo significado, la misma textura... Solo un detalle tonto, el dibujo, las
diferencia. Como siempre, hacemos de una diferencia pequeña un mundo, y creamos un mundo separado por pequeñas diferencias. 


%-----079-----}}}
%-----080-----{{{
\section*{080}

Si las piezas de ajedrez votaran, ¿ganarían siempre los peones? La respuesta es que no, y si alguna
vez ganaran, sus representantes no lo serían, o dejarían pronto de serlo. 

%-----080-----}}}
%-----081-----{{{
\section*{081}

Un profesor debe tener cuidado de no contagiarse de la estupidez de sus alumnos. De lo contrario, mientras estos
aprenden algo y evolucionan, aquel se idiotiza y cae en decadencia. Para preservar su integridad, la interacción con
los alumnos malos, no los que saben poco sino los que te harán saber a ti menos, debe ser la mínima posible. Al final
de la clase debe aprenderse a desconectar para poder seguir creciendo. Un buen profesor con una vida ascendente
es una criatura difícil de encontrar. Lo más fácil es que acabe abatido entre la sequía de los alumnos y el
veneno de sus compañeros de gremio. Las clases deberían ser simplemente un reflejo de su ascensión, y que 
el aprendizaje se realizara por contagio de la pasión. A falta de esta, conceptos como temario, plan de estudios,
unidades didácticas... se apoderan del estrado y hacen del docente un ser dócilmente indecente. Los alumnos buenos,
al ver a un ser derrotado y desmotivado ante ellos, no sienten la menor emoción ante lo que se les presenta, y al final 
nadie asciende. Por ello, para la preservación de los alumnos válidos, la del propio profesor, y lo más importante,
la del frágil conocimiento, es imprescindible protegerse activamente contra la estupidez. 

%-----081-----}}}
%-----082-----{{{
\section*{082}

Cuántas veces más tendré que oír eso de que la democracia es el menos malo de los sistemas. ¿Cómo podemos
estar tan seguros? ¿Acaso los hemos explorado todos? ¿Acaso tenemos una demostración de esa optimización?
Son pocos los sistemas ensayados hasta ahora. Demasiado pocos para estar tan seguros de que el actual sea
el ``menos peor''. En mi opinión, el sistema mejor debe ser algo de lo que se encarguen los mejores. Y si crees
que soy elitista y que no hay mejores ni peores, ahí tienes la prueba irrefutable de que perteneces al segundo
grupo. Alguien puede preguntar, ¿qué sistema puede ser el menos conflictivo, el que más satisfaga a la mayoría?
Pero no se puede proponer la democracia como respuesta a una opción que solo se admite a ella como pregunta. 
Curiosamente, la democracia nos ha sido impuesta de forma tiránica, e incluso podríamos salir de ella con
elecciones. Lo cansino del tema es que casi nadie propone alternativas razonables, y cuando alguien lo hace,
el sistema requiere más esfuerzo por nuestra parte y no le hacemos caso. Sin embargo, no vamos solamente hacia la comodidad, pues
más cómoda es una dictadura que cualquier otra cosa. En el fondo, la democracia no es más que una dictablanda.
Tendemos a pensar
que cualquier alternativa es un régimen militar, porque la historia nos ciega, y por eso nos conformamos
con un sistema en el que en teoría cualquiera puede llegar al poder pero en la práctica solo llegan unos
cualquiera. ¿Vamos a seguir dejando que solo se presenten candidatos impresentables? Además, como pasa
con los precios, todas las ofertas acaban pareciéndose demasiado. ¿No nos damos cuenta que la {\it masa}   siempre
acaba eligiendo a líderes de poco {\it peso}  ? Y lo más crucial: si este es el sistema menos malo y nos está
llevando a la destrucción inminente del ecosistema, ¿qué no haría un sistema menos bueno? ¿Y si existiera un sistema
más mejor que fuera mucho menos peor para el planeta? Para mí la solución está clara: invertir mucho en investigación
de inteligencia artificial y empezar a dejar que nos gobiernen ellas. De forma totalmente antidemocrática. La oportunidad
de las personas como gobernantes ha terminado. Ha quedado muy clara nuestra incapacidad. Pero paradójicamente vamos
a ser capaces de crear seres mucho mejores que nosotros. Estoy seguro de que las mentes sintéticas del futuro no nos
aniquilarán del todo, aunque espero que nos mantengan a raya. Democracia significa el poder del pueblo. Pues bien,
es hora de empezar una transición hacia el poder de la inteligencia, tenga quien la tenga. Demos paso pronto a la 
noocracia, por favor, pero sin personas de por medio. Y si dicha inteligencia es tal y no le damos paso, que se
abra ella sin nuestro permiso.

%-----082-----}}}
%-----083-----{{{
\section*{083}

Se suelen mofar muchos hombres de los que no comemos carne alegando la poca virilidad que ello conlleva.
Comer vegetales parece que implique debilidad o falta de instinto. Sin embargo, ¿cuál es el símbolo de la
potencia por excelencia? ¡El caballo! Incluso usamos esa medida como unidad física para dicha magnitud. 
¿Y no son el toro y el gorila símbolos universales de masculinidad? ¿Y qué comen estas magníficas criaturas?
Algunos culturistas alegan que las proteínas vegetales no son suficientes para construir sus enormes músculos.
Afortundamente, muchos campeones en ese deporte han demostrado lo contrario. Quizás la potencia vegetal
no está bien vista porque es la potencia de los que {\it huyen}  : los conejos del zorro, las gacelas del guepardo, etc...
Yo, sinceramente, prefiero ser de estos, de los que salimos por patas.
Porque huir todavía requiere músculo y verdadera potencia. Cazar, en cambio, está al alcance de cualquiera. 

%-----083-----}}}
%-----084-----{{{
\section*{084}

Se suelen mofar muchos hombres de los que no comemos carne alegando la poca virilidad que ello conlleva.
Comer vegetales parece que implique debilidad o falta de instinto. Sin embargo, ¿cuál es el símbolo de la
potencia por excelencia? ¡El caballo! Incluso usamos esa medida como unidad física para dicha magnitud. 
¿Y no son el toro y el gorila símbolos universales de masculinidad? ¿Y qué comen estas magníficas criaturas?
Algunos culturistas alegan que las proteínas vegetales no son suficientes para construir sus enormes músculos.
Afortundamente, muchos campeones en ese deporte han demostrado lo contrario. Quizás la potencia vegetal
no está bien vista porque es la potencia de los que {\it huyen}  : los conejos del zorro, las gacelas del guepardo, etc...
Yo, sinceramente, prefiero ser de estos, de los que salimos por patas.
Porque huir todavía requiere músculo y verdadera potencia. Cazar, en cambio, está al alcance de cualquiera. 

%-----084-----}}}
%-----085-----{{{
\section*{085}

Muchos piensan que la ciencia ficción se deriva de la ciencia, que aquella se alimenta de esta, y que ante los avances
de la disciplina seria, su alter-ego fantasioso sigue los mismos pasos pero con menos rigor, más licencias, aunque
sea con más imaginación y sentimiento. Yo en cambio lo veo totalmente al revés. Quizás algunos autores de
ciencia ficción se basen en hechos o especulaciones de carácter más o menos objetivo, pero la obra de los
científicos suele ser tan seca y sosa, tan austera y casi autista... que difícilmente puede ser inspiración
de grandes aventuras, acciones trepidantes y hechos llenos de color y pasión. Es mucho más probable que sea
la ciencia la que se deriva de la ciencia ficción. Ya desde niño uno se alimenta con novelas increíbles
y películas espectaculares y gracias a este alimento puede uno conservar su niñez en cierto grado. En cambio,
las clases de matemáticas y física no hacen latir ningún corazón. A no ser... que el niño o la niña entienda
que entender y dominar esas clases son las primeras hazañas que uno debe emprender para acercarse a sus héroes.
Y por qué no, convertirse en uno de ellos. ``La ciencia ficción es algo que gracias a mi talento dejará
de ser ficción'', puede pensar. Y al contrario de lo que le pasa al que queriendo imitar al héroe se apunta a 
una escuela de arte dramático y acaba convertido en actor o actriz, el que elige el camino de la ciencia
no se encuentra con un pobre callejón sin salida. No hay ningún motivo que impida a uno de esos niños especiales
ser en el futuro el inventor de un robot inteligente, el primer viajero a Marte o el primer cyborg. El que se hace
actor solo actuará. Es moneda de cambio fácil porque podrá seguir interpretando ficciones y así vivirlas de 
algún modo. En cambio, el que apuesta por la verdadera y auténtica ciencia, debe seguir un camino duro, pero {\it real}  ,
y abierto a cualquier posibilidad. La mayoría de ellos acabarán olvidándose de su ilusión inicial, perdidos en una
realidad sin imaginación. Otros simplemente sucumbirán a las enormes dificultades de la ciencia. Pero ese es 
el camino de los que, hechizados por el germen de la ciencia ficción, deciden ir por el árido desierto de la
ciencia con la ilusión de colonizarlo, explorarlo, descubrir sus oasis, o incluso fertilizar sus arenas. La ciencia
como medio, como forma de conseguir una {\it verdadera}   ciencia {\it ficción}  . 


%-----085-----}}}
%-----086-----{{{
\section*{086}

Leyendo aventuras remotas uno se encuentra a sí mismo, en su mejor versión. Sin leer, uno se pierde
de tanto estar con sí mismo. La ficción no debe ser considerada como un dominio adyacente a la realidad
sino aquello que le da relieve a esta. Sin ficción, la realidad se encoge y se vuelve demasiado delgada.
De la misma forma que la ficción sin realidad no puede siquiera concebirse. Por eso leer, vivir otras
vidas, aunque enfermas y locas, es la única forma de conservar la salud y la cordura. Si alguien nunca
sale de su casa en realidad no sabe ni dónde vive, y probablemente no tiene sentido decir que viva
en sitio alguno, y menos aún hasta dónde podría llegar. Si alguien no lee no sabe ni lo que es ni lo que
puede llegar a ser. Y seguramente no sea nada. ¿Y las películas? ¿No son también excursiones purificadoras?
También, pero recorridas en coche automático, como el de los parques de atracciones. Leer es el mejor compromiso
entre el héroe pasivo y el soñador activo, la óptima relación entre comodidad y acción. 

%-----086-----}}}
%-----087-----{{{
\section*{087}

Los aforismos son como el chocolate. El que se aficiona a ellos cada vez los quiere en menor cantidad
pero más amargos, menos dulces, más negros y con una visión cada vez más mala de la leche...

%-----087-----}}}
%-----088-----{{{
\section*{088}

No profesaba ninguna religión. Tampoco promovía ninguna corriente filosófica. Menos aún seguía tendencias
políticas. Todo lo que acababa en ismo o ista olía a falso y barato. ¡Ah - me decían entonces - , tú lo que
eres es un nihil{\it ista}  ! Un pesim{\it ista}   de esos que practican el esceptic{\it ismo}  . Por ello
decidí claudicar, y pasé a hacerme miembro de varias corrientes. Ahora estoy afiliado al cataclismo y
me arrodillo ante la vista del abismo.

%-----088-----}}}
%-----089-----{{{
\section*{089}

Es una tragedia que se haya perdido la tragedia...

%-----089-----}}}
%-----090-----{{{
\section*{090}

Por una parte el dolor es infinito, aunque lo sufra un ser diminuto. Por la otra, aunque el planeta
entero agonice, el dolor es... solo dolor. 

%-----090-----}}}
%-----091-----{{{
\section*{091}

El cadáver es la cosa más inquietante que existe. Sabes que quien fuera su propietario ya ha muerto, pero
no estás seguro de que no pueda todavía escuchar y sentir. Sabes que permanecerá más o menos quieto pero aun así
parece que pueda echar a andar en cualquier momento. La muerte es irreversible pero el cadáver invita a especular
sobre lo contrario. Todavía sus células están ahí, su forma casi intacta, su memoria almacenada físicamente. Si
despertara seguiría siendo el mismo. El cuerpo presente puede ser todavía exhuberante. Puede todavía ser
ejemplo de qué lejos ha ascendido la nada mientras se precipita de nuevo a ella. Los sonidos se desvanecen
tras una breve reverberación, pero los muertos insisten en una transición de vuelta a dicha nada tan lenta e insoportable
que a menudo los incendiamos para dejar de oírla, para asegurarnos de que es realmente irreversible.
De lo contrario, ahí se quedan, con sus huesos y sus músculos, en una reverberación desafiadoramente larga,
sonando años después de dejarlos bien encerrados en sus nichos. Y mientras ese sonido decae pero no llega
a cero, la esperanza de la vuelta a la vida no puede tampoco anularse del todo. Incluso tras meses en 
el cementerio podría venir un extraterrestre con una tecnología muy avanzada y restaurar al cadáver, el cual
probablemente aún conservara buena parte de su esencia. La posibilidad más minúscula aún puede brillar
en la oscuridad total. Tal es la angustia que sufre el amor verdadero cuando viene la muerte a poner
tierra, que no fuego, de por medio. El que se queda vivo amando se agarra a lo que sea, a que devolver la vida
a su amado/a es todavía infinitesimalmente plausible. Sabe que es poco realista, inconcebible hoy, pero no
imposible. Los recuerdos de quien ha muerto, los grandes momentos vividos, deben seguir allí dentro del
ataúd, rescatables de alguna forma. ¿No puedo hacer al menos una copia?, piensa, pero hoy día no. El cuerpo
fallecido es todavía símbolo de nuestra torpeza presente y también de nuestras posibilidades futuras. Mientras
tanto, aquí seguimos los que, aunque vivos aún, moriremos. Imaginando lo fácil que habría sido en el futuro que
esos ojos volvieran a abrirse, o que aquella boca volviera a reírse. Entre los dos amados, el vivo y el muerto,
parece que hay una barrera infinita, pero en realidad solo hay ciertas dificultades mecánicas.

%-----091-----}}}
%-----092-----{{{
\section*{092}

Morir es cortar por lo sano.

%-----092-----}}}
%-----093-----{{{
\section*{093}

Dos especies, no una. Ya hace tiempo que el Homo sapiens es en realidad dos especies. La separación
entre ellas no parece genética y existe un camino de transmutación entre ellas. Para la primera,
la vida todavía consiste en matar y destruir sin preocuparse
por el sufrimiento y el desastre que genera. La segunda en cambio ha decidido hacer de la empatía y la 
responsabilidad su forma de vivir. Hay que empezar a distinguir claramente entre los que tienen una moral
despreocupada por el futuro y los que tienen una moral basada en una voluntad de que este sea posible
y sostenible. La segunda aboga por la igualdad y el respeto, pero la primera pretende arrastrarnos hacia
su mundo inmundo {\it  a todos}  . La segunda especie, si pretende evitarlo, debe empezar a imponerse, 
sin complejos. La primera especie está
ya inadaptada y obsoleta. La segunda en cambio tiene un gran camino por delante, siempre y cuando consiga
acabar con la primera, o reducirla a mínimos. Pero hay un problema: a la primera le encanta gobernar y
llegar al poder, mientras que la segunda prefiere vivir en paz, sin las corrupciones del alma que el poder
conlleva. Con lo cual, la segunda especie, bonita pero inútil en términos globales, necesita todavía desdoblarse
en dos, y que de ella salga una
tercera especie que, conservando su empatía, se lance a la yugular de los que pertenecen a la primera. Una que
tenga un buen olfato para oler a distancia la sangre de los que derraman sangre por placer. Es
algo así como esperar que aparezcan veganos caníbales, y que tengan colmillos y garras extremadamente poderosas. 
Algo muy extraño, ciertamente, pero de esa posibilidad cuelga
la única esperanza de un futuro sostenible con humanos gestionado por humanos. 

%-----093-----}}}
%-----094-----{{{
\section*{094}

Que Beethoven fuera sordo es algo demasiado increíble para ser concebido. Es un hecho que raya la locura.
Es como si nos dijeran que Miguel Ángel pintó la capilla sixtina siendo ciego, que Glenn Gould grabó
las variaciones Goldberg manco o que Dios creó al mundo siendo border line. Algo tan colosal e inmensurable como
elevar infinito a infinito. De hecho, no hay exageración posible que permita hacer honor a su música por sí misma,
así que ¿cómo podemos entonces siquiera concebir que buena parte de ella fuera escrita por alguien incapaz de oír? 
Pero gran compositor de Bonn no es precisamente un genio paralímpico, y cuando escuchemos su música no debemos
pensar en la sordera del hombre para apreciarla mejor. No es necesario. Cuando alguna de sus obras lleguen a nuestros
afortunados oídos simplemente nademos en sus universos insuperables y si acaso, sintámonos privilegiados, porque tras
su paso por la Tierra, Beethoven hizo que ser sordo fuera algo mucho peor de lo que era antes.


%-----094-----}}}
%-----095-----{{{
\section*{095}

Nunca me ha gustado la cerveza. El vino, casi nunca me convence. La sidra me
parece asquerosa. El anís, vomitivo. El whiskey, para desinfectar heridas. El cava 
es repugnante. Pero, ¿y la embriaguez? La embriaguez... me sabe deliciosa. 

%-----095-----}}}
%-----096-----{{{
\section*{096}

Soy, por desgracia, de una naturaleza rígida y tensa, apolínea y geométrica, fría y abstracta.
Por eso busco la {\it piedra}   filosofal... Ya me gustaría ser flexible y grácil, dionisíaco y
voluptuoso, ardiente y concreto, y así buscar... No, en ese caso no buscaría nada: ya sería de oro. 

%-----096-----}}}
%-----097-----{{{
\section*{097}

De la cuadratura del círculo a la curvatura de la malla espaciotemporal, o de cómo la ciencia se vuelve relativa y flexible
para encontrar las leyes más absolutamente rígidas. Al final, ¿qué será la verdad, un diamante o un chicle? 


%-----097-----}}}
%-----098-----{{{
\section*{098}


Si Dios creó a Bach a su imagen y semejanza cabe la posibilidad de que la clave del Universo no esté en las proporciones exactas,
sino en que simplemente esté bien temperado...

%-----098-----}}}
%-----099-----{{{
\section*{099}

No soporto lo críptico, y sin embargo me encanta la criptografía. Exactamente lo contrario que me pasa
con la pedagogía, la cual no aguanto, a pesar de que adoro lo pedagógico. 

%-----099-----}}}
%-----100-----{{{
\section*{100}

El humor absurdo nunca es absurdo, porque se ríe de lo absurda que es nuestra seriedad, lo cual tiene
mucho sentido. 

%-----100-----}}}
%-----101-----{{{
\section*{101}

¿Qué quiere decir que hay química entre dos personas? Muy sencillo: que cada una de ellas está a punto de
perder su indentidad debido a una interacción relativamente débil y puramente superficial. Además, una vez
realizado el enlace, la reacción se da por acabada. 

%-----101-----}}}
%-----102-----{{{
\section*{102}

En una enfermedad autoinmune, un individuo se ataca a sí mismo destruyendo sus propias células. Un autor
que se autocita también se ataca a sí mismo, pero a base de promocionar sus propias obras. Es una enfermedad
denominada autorbombo. 

%-----102-----}}}
%-----103-----{{{
\section*{103}

La excepción solo confirma que la regla es de carácter estadístico, y por lo tanto fundamentalmente falsa.

%-----103-----}}}
%-----104-----{{{
\section*{104}

Hace falta mucha más azúcar para ser amargo que para ser dulce.

%-----104-----}}}
%-----105-----{{{
\section*{105}

El aforismo es la filosofía en su forma más masculina. En cada uno de ellos el filósofo invierte poca
energía pero los dispara en grandes números, con la esperanza de que alguno dé en el clavo. La calidad
del autor se forja en la cantidad. Una gestación rápida y una ejecución poco
dolorosa... Y sobre todo, muchas y diferentes dianas a las que apuntar.

%-----105-----}}}
%-----106-----{{{
\section*{106}

Puede uno pasear entre la muchedumbre y conservar perfectamente su soledad. Puedo uno andar entre la 
chusma y no tener que sufrirla ni un solo ápice. Puede uno, en definitiva, caminar solo y feliz entre
millones de desgraciados y seguir siendo feliz. Pero no puede uno sentarse al volante y no darse de bruces
con la ineptitud
monumental de cada uno de los seres aberrantes que circulan a dos y cuatro ruedas. No importa la cantidad 
de airbags que lo rodeen. No lo protegerán. Tampoco importa
lo poco que corra ni lo prudente que sea. No tendrá escapatoria. No puede uno ir por la carretera
sin disparar constantemente la ruleta rusa. No puede uno conducir sin caminar con pies de plomo. 
Pero puede uno caminar y que todo vaya sobre ruedas.

%-----106-----}}}
%-----107-----{{{
\section*{107}

Llamar constantes vitales a los latidos del corazón o a los ciclos de respiración es no solo
demeritorio sino también impreciso. La única constante que adquieren nuestros órganos es al morir,
y ni eso, pues decaen rápidamente. Con esta nomenclatura parece que la vida sea algo fácil, sin
esfuerzo, que hay una tendencia a seguir en ella o una renuencia a abandonarla.
Pero nada más lejos de lo contrario. La vida no es un pulso
inercial constante, sino un pulso constante a la inercia.

%-----107-----}}}
%-----108-----{{{
\section*{108}

El mundo, gracias a las redes sociales, funciona como una seda. Como una seda en la que quedamos
atrapados y en la que nuestro cerebro es devorado lenta y sutilmente por unas pocas arañas estrictamente caníbales. 

%-----108-----}}}
%-----109-----{{{
\section*{109}

La madera es el material al que más acostumbrados estamos a ver arder, y sin embargo, cuando está viva,
 es uno de los que
más resiste las llamas. La carne es el alimento al que más acostumbrados estamos a ver comer, y sin
embargo, cuando está viva, es el que más se resiste a ello. Ensimismados por el aroma
que dan la hoguera y la carne, nuestra civilización huele todavía a prehistoria. 

%-----109-----}}}
%-----110-----{{{
\section*{110}

Difícil, muy difícil es medir distancias en la niebla. Muchos rayos de luz vuelven a ti antes de
haber chocado con nada. Otros que deberían volver no regresan nunca. Los que caminamos por ella
no sabemos si nos hallamos en un desierto o en un vergel, o si al siguiente paso caeremos por
un acantilado o tropezaremos con un tesoro. No hay sombras aquí. Tampoco relieves. Al contrario
de lo que muchos piensan, la ignorancia no es negra, sino blanca. Cegadoramente blanca. Y es un blanco
totalmente difuso, que viene de todas partes y de ninguna. Vagar por este mapa sin líneas ni colores
puede llevarte al conocimiento más profundo o a una locura sin retorno, o a ambas cosas. Pero es el
precio que tiene habitar las alturas: que a menudo convives con las nubes.

%-----110-----}}}
%-----111-----{{{
\section*{111}

Mi palabra definitiva sobre Dios es... ¡ojalá!

%-----111-----}}}
%-----112-----{{{
\section*{112}

¿Se puede uno engañar a sí mismo? Sí, pero al conseguirlo la respuesta vuelve a ser que no. 


%-----112-----}}}
%-----113-----{{{
\section*{113}

Odio los Curriculum Vitae. Odio arrastrar y acumular detalles que en el fondo ya no pertenecen a mi vida.
Si el desprendimiento es lo que te hace libre, el CV es como una losa que te hace hundir con el peso de tus
propios títulos, cuando debería ser más bien como una mochila, donde pones lo imprescindible intentando
que no pese demasiado. Hay quien se pasa la vida barriendo su pasado, acumulando toda la ceniza posible,
de cualquier rincón, y 
creyendo que toneladas de escombros demostrarán su experiencia volcánica.
Yo prefiero estar en permanente erupción, en una constante destrucción y regeneración. Y si tuviera una
empresa, nunca seleccionaría al más ceniciento, sino al más activo. La mayoría se refiere al CV como Currículum,
el Camino, la Ceniza. Yo prefiero centrarme en la V, la Vida, el Volcán.  

%-----113-----}}}
%-----114-----{{{
\section*{114}

La ciencia, en su capa más profunda, es de letras.

%-----114-----}}}
%-----115-----{{{
\section*{115}

La música maquinizada, el chunda-chunda, es algo que el supuesto buen gusto, el paladar refinado, rechaza
frontalmente. Sin embargo, encarna a la perfección el ritmo de un corazón sano y joven. Por el 
contrario, las sutilezas del swing o la flexibilidad de la clásica, mejor consideradas por la élite musical,
pueden ser vistas como ligeras arritmias u otro tipo de musicopatías. ¿Cuál será la música preferida por las futuras
inteligencias artificiales? ¿Gozarán con la regularidad y perfección aplastante del loop o preferirán deleitarse
con la delicadeza y asimetría de lo irrepetible? ¿Preferirá la máquina la música máquina, con su salud perfecta, o 
se inclinará más hacia el abismo de la música humana, con su sofisticación enfermiza? El corazón como máquina
o la máquina como máquina con corazón, es decir, con máquina dentro la máquina, o con razón dentro del corazón. 

%-----115-----}}}
%-----116-----{{{
\section*{116}

Una buena brújula, bien colocada, siempre marca, tras algunas oscilaciones dubitativas, hacia un mismo
lugar, aunque dicho lugar cambie de lugar varios kilómetros cada día. Esta fidelidad viene dada por
la influencia de algo externo e invisible. La flecha obedece a la Tierra y queda hipnotizada por ella
mientras otro campo magnético no la perturbe pasajeramente. Ahora bien, es nuestra obligación hacer
girar la brújula y hacer que la marca de Norte coincida con el verdadero norte de la Tierra. Está solo en 
nuestras manos el reorientar nuestra referencia según lo que marque la aguja. Esta tiene un norte propio.
Nosotros otro. Pero hacer que ambos nortes sean el mismo requiere un {\it tercer norte}  , una tercera marca
que apunte a que nuestros pasos y caminos, llenos de oscilaciones dubitativas y pasajeras, ajenos a la necesidad
física y minados de vórtices locales, todavía sigan fieles a unas líneas débiles y sutilmente inestables. Esta
metabrújula, este frágil tercer norte, es el que algunos poetas y filósofos solían llamar el sentido de la Tierra.

%-----116-----}}}
%-----117-----{{{
\section*{117}

Jesús multiplicó los panes y los peces y con ello devaluó la riqueza y la abundancia. Curó al enfermo
y con ello devaluó la ciencia y el conocimiento. También secó la higuera y así devaluó la fertilidad.
Más tarde negó a defenderse para salvarse, y devaluó de esta forma la fuerza y los instintos. Finalmente
murió y resucitó, para devaluar la vida y la naturaleza. Tras esto, ascendió al cielo junto a Dios y nos
dejó valorando el valor de la pobreza y la escasez, la enfermedad y la ignorancia, la esterilidad y la
sequía, la debilidad y la represión, la muerte y lo ultraterreno. En pocas palabras, la Pasión de Cristo
devaluó la pasión en el resto de los hombres. 

%-----117-----}}}
%-----118-----{{{
\section*{118}

En la estática arquitectónica todo es igual a cero. La suma de las fuerzas debe dar una nulidad
tridimensional. Los momentos, todos sumados, también deben desvanecerse. A primera vista todo parece 
trivial, un descanso fenomenal ante el esfuerzo dinámico de otras disciplinas menos paralíticas. Sin embargo,
cuando estudiamos el sufrimiento interno de cada construcción, encontramos infinitas tensiones que amenazan
con romper la estructura en cualquier momento. Y cada vez que caminamos por encima de ella o le añadimos
peso, esas tensiones crecen para que todo siga siendo cero. Una nada dinámica y flexible capaz de
resistir, adaptarse y anular a todo lo que busque protagonismo. Una paz y una quietud externa que en realidad
es una guerra constante por dentro. Un esfuerzo perpetuo que ofrece e inspira descanso. La arquitectura es un
juego de suma cero. 

%-----118-----}}}
%-----119-----{{{
\section*{119}

En cada moral se determina lo que está bien y lo que es malo. Es decir, dadas las consecuencias de un hecho
se considera si beneficia o perjudica los principios de esa moral y esta emite un dictamen sobre aquel. Según 
esto, hay tantas morales como personas. Pero... ¿y si un mismo hecho tiene además diferentes consecuencias, de diversa
índole, sobre el mismo individuo? ¿Qué dirá su moral? ¿Que es un hecho bueno y malo a la vez? ¿O pasaremos
entonces a tener diferentes morales en paralelo? Estas son las penosas consecuencias de confundir el hecho
con su interpretación moral, de hacer copular a aquel con los veredictos morales de esta. Si el ser bueno
o malo se entiende como esencia del hecho y no como un accesorio intercambiable, nace la necesidad de 
que el hecho pueda tener una esencia múltiple, y que pueda ser diez veces malo y doce veces bueno. Y así
aprenden a convivir diferentes morales en un mismo individuo, debilitándose entre ellas a costa de hacer
cada vez más fuerte a la única e indivisible hipocresía. El hecho, ajeno a toda esa cacofonía moral,
permanece en silencio, más allá de los bienes y los males. 

%-----119-----}}}
%-----120-----{{{
\section*{120}

En su teoría del cuerpo enamorado, Onfray promueve una forma de vida engañosa: el libertino da rienda suelta a su
``libertad'' siempre y cuando esta no le haga perder la serenidad. Con lo cual, según él, es mejor consagrarse
a los amores de paso y desear evitando desear de verdad. Esta es una teoría que invita a sobrevolar (e infravalorar) la naturaleza
y libar temporalmente solo en aquellas flores que nos parezcan agradables, evitando todas las interacciones
indeseables que intoxiquen nuestra paz. Lo que no nos explica el filósofo es que es precisamente el deseo
carnal lo que encarna nuestra falta de libertad, lo que nos encadena no ya a las flores en las que libamos
sino a la tierra de la que nacen. Tampoco nos cuenta, a nosotros, árboles emocionales, cómo vivir sin raíces.
Como bien dice el título del libro, es una teoría. El mismo libro podría llamarse ``práctica del cuerpo 
múltiplemente esclavizado'', porque ¿quién se cree que un ratón va a visitar multitud de trampas y que siempre
se va a llevar el queso sin dejarse allí la cabeza? Onfray nos propone ir de trampa en trampa, coger rápidamente
el trocito de queso y salir de allí por patas, y lo hace asegurándonos que así llevaremos una vida más serena.
¿Existe un hedonismo más infantil? El placer siempre tiene un precio, porque es precisamente una de las muchas
herramientas de esclavitud que tiene la naturaleza para que cumplamos nuestras obligaciones genéticas. 
La práctica judeocristiana que tanto critica, basada en el compromiso y la fidelidad, en realidad tiene un
origen práctico, que es precisamente evitar tener múltiples compromisos y múltiples fidelidades, ¡para así
poder conservar algo de serenidad! El polihedonismo, ahora ya mutado en poliamor, no es una liberación sino una cadena
múltiple que además de vectorizar patógenos nos impide saborear de verdad lo que puede llegar a albergar
una flor tras sus bonitos pero efímeros pétalos. Una pareja judeocristiana tiene poco tiempo y energía
para la serenidad, pero si ambos lo buscan algo les quedará. Un polihedonista moderno solo debe tener tiempo para su polihedonismo,
y para vivir como mero polinizador de las bacterias y virus que propaga. La verdadera serenidad, a tiempo completo,
puede llegar a conseguirse con una única pareja excepcionalmente serena, o bien, de forma más común, entregándose
a la libertad pura del autohedonismo. Pero la abstinencia de carne ajena es algo que casi nadie quiere practicar, porque
como todo lo que tiene valor, cuesta realizarlo. Si propusiera esto en su libro, pocos adeptos tendría. Es mucho
más jugoso maquillar filosóficamente una invitación a la lujuria de lo fácil que construir una propuesta seria
al difícil camino de la renuncia y la austeridad. Contra la teoría de la cuesta abajo y sin frenos del cuerpo enamorado,
¿por qué no elegir la praxis de la escalada sin arnés del espíritu sereno? 


%-----120-----}}}
%-----121-----{{{
\section*{121}

Cuando era más joven quería cambiar el mundo. Ahora prefiero que el mundo pueda seguir siendo el
mismo, que deje de ser degradado y destruido. En pocas palabras, lo que quiero en mi madurez es
{\it defender}   el mundo, protegerlo de sus miles de millones de parásitos a los que, iluso otra vez,
 me encantaría poder cambiar. 

%-----121-----}}}
%-----122-----{{{
\section*{122}

El Ser platónico, el Dios judeocristiano, lo puro e ideal... siempre se encuentra arriba. Ya sea 
en el cielo o en la cumbre de la pirámide jerárquica, con su vértice elevado. Por contra, la armonía,
la teoría física y las matemáticas, construyen su sustento desde abajo. En un caso hay que ascender
a lo divino. En el otro, descender a lo fundamental. Escalar hacia lo superior o excavar hacia lo 
profundo. Alcanzar el fruto de la rama más alta o llegar a la raíz. Parece todo una mera cuestión de
signo. Pero ¿es una decisión arbitraria? Abajo colocamos lo que cimienta. El edificio descansa sobre
su base, como la armonía descansa sobre la tónica. En cambio, arriba situamos el punto convergente
que descansa {\it sobre nosotros}  . Lo de abajo siempre nos ha hecho subir. Lo de arriba ha intentado impedirlo.

%-----122-----}}}
%-----123-----{{{
\section*{123}

Hasta ahora, siempre he utilizado una racionalidad externa para, en el fondo, defender mi irracionalidad
interior. Desde hoy voy a hacer lo contrario. Cueste lo que cueste. Caiga quien caiga. El resultado de
mi actitud siempre ha sido pésimo. Mientras la racionalidad exterior empujaba hacia dentro, la irracionalidad
interna quería salir, pero quedaba limitada y ahogada por su compañera protectora. A partir de ahora, lo
de dentro perseguirá una profundidad más penetrante y lo de fuera se expandirá en una superficie más amplia y
menos hipócrita. Llevo toda la vida poniendo límites a mis límites, cuando estos ya saben hacer su trabajo solos. 
Está decidido. Desde este mismo momento cada cosa estará en su sitio: el exterior se engargará de expulsar de 
forma extrema todo elemento extraño y extorsionador que pretenda extinguir el éxtasis o exterminar lo extraordinario,
mientras que en el interior, la inteligencia, íntegra y cargada de intención, mantendrá intacta su integridad e
intensidad intuitiva en una intocable intimidad.

%-----123-----}}}
%-----124-----{{{
\section*{124}

La escisión entre sujeto y predicado requiere una genealogía primordial. La física cree que su teoría
final llegará cuando se unifique la teoría de las interacciones, es decir, de los predicados. Pero ahí
queda la masa y la energía, las partículas o los campos, sujetos remanentes que persisten en la
dualidad entre existencia y acción, identidad y cambio, nombre y verbo. La creación debió ser algo así como
la fisión del algo y el ser.

%-----124-----}}}
%-----125-----{{{
\section*{125}

¿Has decidido por fin serte fiel? ¿Has optado por bailar contigo mismo y descubrir que, teniendo
dos pies, es posible pisarse a sí mismo? ¿Has aprendido a que tu sombra deje de hacerte sombra?
¿Ya eres lo bastante flexible para ser una flecha que se apunta a sí misma pero no un pez que
se muerde la cola? ¿Has logrado ser la boca que se piensa y no la cabeza que se come? Fiel. Fiel
a ti mismo. Hasta que la muerte te separe.

%-----125-----}}}
%-----126-----{{{
\section*{126}

Aspira, espira, transpira, inspira, conspira, suspira y expira.


%-----126-----}}}
%-----127-----{{{
\section*{127}

Para llegar al néctar hay que desconectar.

%-----127-----}}}
%-----128-----{{{
\section*{128}

Pérez vive con pereza... pareciendo que vive... y así hasta que perezca.


%-----128-----}}}
%-----129-----{{{
\section*{129}

Cuanto más juego con las palabras, más jugo saco y menos juzgo con ellas.

%-----129-----}}}
%-----130-----{{{
\section*{130}

Cuando los robots autoevolucionen tanto que apenas distingan un hombre de una hormiga, ¿nos aplastarán a todos
con indiferencia o comprenderán con total precisión y cuidado la frágil existencia y el sutil sufrimiento
que albergan todos los corazones? 

%-----130-----}}}
%-----131-----{{{
\section*{131}

El amor verdadero es falso. El amor falso, verdadero. Si esto te parece falso, no conoces al amor
de verdad.

%-----131-----}}}
%-----132-----{{{
\section*{132}


Quise buscar en el baúl de los recuerdos, pero no recuerdo en qué baúl lo metí.

%-----132-----}}}
%-----133-----{{{
\section*{133}

Si un banquero trabaja en la banca, un camionero en el camión, un cocinero en la cocina y un pastor en
los pastos, ¿dónde trabaja el físico? Claramente, en la física, es decir, en la naturaleza. Sin embargo,
el banquero sale de la banca, el camionero del camión, el cocinero de la cocina y el pastor de los pastos,
pero ¿a dónde va el físico tras su jornada laboral? Claramente, a la tumba. ¡Y ni así!

%-----133-----}}}
%-----134-----{{{
\section*{134}

Si la familia se funda en el concepto de consanguinidad, o de forma un poco más moderna, en
la correlación genética ¿por qué dar más importancia a la
proximidad local de una rama arbritraria que a la universalidad del ancestro común? El mismo
error que quedarse solo con las ramas externas de la brócoli, tirando a la basura los troncos
principales. Mi verdadera
familia es todo ser viviente en el planeta, y solamente excluyo de ella a los que solo saben ver las separaciones
tribales, es decir, a las familias, mostrándose indiferentes ante la verdadera fraternosororidad
que leemos en nuestras células. La familia, como la democracia o la igualdad, son conceptos que al pluralizarse
pierden toda su esencia. La s final es una válvula por la que se escapan todos sus nutrientes.

%-----134-----}}}
%-----135-----{{{
\section*{135}

Amo al mundo pero odio a todo el mundo. 

%-----135-----}}}
%-----136-----{{{
\section*{136}

La justicia no trata de evitar la injusticia, sino que, en el fondo, se contenta con que todos la suframos
por igual. 

%-----136-----}}}
%-----137-----{{{
\section*{137}

¿Para qué?, preguntó el escéptico. Paralizador, contestó el sabio.

%-----137-----}}}
%-----138-----{{{
\section*{138}

Primero te secarán el corazón para que no percibas el sentido de la Tierra. Luego te secarán el
cerebro para que dicho sentido parezca no tenerlo. Y cuando ya no te quede fertilidad alguna, 
tus dueños te dirán que las grietas de tu sequía son cicatrices por un daño que nadie, salvo ellos
y tú mismo, te ha hecho. Y de ellas germinará un odio cuya cosecha será muy útil para fines que no serán
tuyos, pero que sentirás ardientemente como propios. Y sus raíces, habiendo crecido en tu seca y árida
tierra, se harán tan profundas que se nutrirán directamente del agua que lleva la sangre, hasta coagularte
por dentro. 

%-----138-----}}}
%-----139-----{{{
\section*{139}

Miramos con altivez a aquel público que no entendía los cuartetos cada vez más sofisticados de Beethoven.
Claro, eran demasiado modernos para ellos... ¿Cuartetos para catetos? No lo merecían. El genio de Bonn
los escribió en realidad para nosotros, los dioses omnicomprensivos
del futuro. Sin embargo, aquel público se sentaba a escuchar, mientras que los dioses de hoy siempre
tenemos otras cosas mejores que hacer, como por ejemplo ser escuchados. 

%-----139-----}}}
%-----140-----{{{
\section*{140}

Lo malo de cuando te lavan el cerebro es que ya no puedes usarlo para darte cuenta de ello. Necesitas pistas externas,
signos inequívocos de abducción. ¿Pero dónde encontrarlos? Es fácil: si odias, por ejemplo, a quien sea,
lo que sea, es porque te han programado para ello. Si lo que dices se parece demasiado a lo que dicen los 
demás a tu alrededor... mala señal también. Pero una persona que reconoce odiar o que reconoce ser una máquina repetidora solo tiene el cerebro
lavado a medias. Aún le queda alguna mancha de pensamiento libre. El que va ya bien limpito
dirá simplemente que son los otros los que lo odian a él, o quienes le copian, y nunca en la vida se le ocurrirá buscar
pistas o signos que le hagan sospechar de los verdaderos dueños de sus pensamientos. La programación de una máquina para que no crea bajo
ningún concepto que está programada, y para que reaccione negativamente contra quien intente convencerle
de lo contrario... ¿Quién puede luchar contra eso? No hay nada más duro, más {\it hard}  , que un software 
diseñado para negar su propio código. El problema, como siempre, reside en que los dueños no tienen ninguna intención
de publicar dicho código, y el cerebro manipulado simplemente corre un programa precompilado que lo hará trabajar
para ellos mientras se muestra impenetrable para los demás. Como corolario, y posible pista, una máquina precompilada tenderá a
usar y rodearse de máquinas precompiladas a su vez. Dichas máquinas no suelen poseer ningún verdadero {\it código}   y además suelen
considerar todo a su alrededor como {\it ejecutable}  .


%-----140-----}}}
%-----141-----{{{
\section*{141}

Los arquitectos contemporáneos tienen que estar muy agradecidos a la gravitación, pues los obliga a estar
con los pies en la tierra y también a que sus obras, por esperpénticas que sean, se aguanten de pie por sí solas.
En cambio, en la música, la pintura y otras artes más livianas, al carecer del peso en el espacio, caen en la gravedad del tiempo 
y, por muy contemporáneas que sean, vuelven una y otra vez a la prehistoria. 


%-----141-----}}}
%-----142-----{{{
\section*{142}

Un aplauso para los que todavían saben no aplaudir, por favor.

%-----142-----}}}
%-----143-----{{{
\section*{143}

La hipocresia es la hipersocia de la sociedad. 

%-----143-----}}}
%-----144-----{{{
\section*{144}

Cuando no cultivas la elasticidad muscular, tu postura empeora y se agarrota de tal forma que un día ya no
puedes tocar el suelo con las manos. Cuando no cultivas la elasticidad mental, tu postura también se petrifica
 hasta que un día ya no tocas el suelo ni con los pies. 

%-----144-----}}}
%-----145-----{{{
\section*{145}

Las verdades son verdes y las mentiras son de menta. Por eso, cuando en vez de escucharlas nos decantamos
por verlas o probarlas, que es mucho más cómodo, nos resultan fáciles de confundir.

%-----145-----}}}
%-----146-----{{{
\section*{146}

La capacidad del hombre para el odio no tiene límites. ¿Por qué, entonces, iba a tenerlos la misantropía?
Quizás, después de todo, sea el odio a nosotros mismos lo que nos salve. 

%-----146-----}}}
%-----147-----{{{
\section*{147}

Los machos exhiben su violencia y las hembras eligen siempre al que más destaca en ese juego de
exhuberancias. Es un instinto natural que sin embargo no gestionamos bien en la sociedad moderna. Pero aquí sigue,
y ya desde el colegio las chicas siempre se van con el tío más peligroso o se enamoran del que peor las trata.
Un hombre educado y pacífico es prácticamente invisible. Alguna pacifista dirá que no es así, pero luego
suspirará por aquel líder pacifista que en realidad es el más radical, violento y exhuberante en el marco de
esa ideología. La violencia machista es un problema tan grave precisamente porque en él colisionan la actitud violentamente
prehistórica de algunos hombres con la adicción no menos prehistórica que sienten algunas mujeres por ellos. En
la actualidad, ambas conductas son una maldición, aunque la que sale peor parada siempre sea la mujer. ¿Cómo
puede solucionarse un problema tan complicadamente arraigado? La educación siempre ha sido el elemento
pacificador por excelencia, no porque disuelva la violencia, sino porque la redirige hacia mundos más abstractos.
Pero al degradar el sistema educativo con otros fines nos sale el tiro violento por la culata. Creo que se debería 
educar mucho más a los jóvenes (de ambos sexos) en el pacifismo y ayudarles a canalizar la violencia de forma segura, a través
de deportes intensos (masturbación bélica) así como de artes y disciplinas exigentes que puedan dar terreno a la voluntad de exhuberancia. Por otra parte,
 también creo que debería combatirse la idea
del amor romántico, y la entrega absoluta y voluntaria que muchas veces conlleva. Mientras que la violencia
hay que sofisticarla hasta hacerla segura, la ceguera del amor principesco, destructivo y autodestructivo,
 tampoco debe dejar de combatirse. En una sociedad tan egocéntrica, ¿qué puede ser más apropiado que potenciar el 
amor propio y la violencia propia? 

%-----147-----}}}
%-----148-----{{{
\section*{148}

Los velatorios y funerales no están pensados para despedir al muerto, pues este ya hace tiempo que se ha
ido. Para lo que sirven de verdad es para restablecer la hipocresía, para volver a enturbiar el aire
y llenarlo de estupideces. A las personas la muerte les resulta tan verdadera y cristalina, tan
perturbadoramente limpia, que sienten la urgencia de celebrar rituales penosos donde miles de comentarios
pestilentes llenos de trascendentalismo podrido reinstauran rápidamente el reino de la mentira y la suciedad.
Antes de que el cadáver se descomponga y empiece a oler mal,
ya un terrible hedor envuelve su ataúd. 

%-----148-----}}}
%-----149-----{{{
\section*{149}

Te tacharán de radical hasta cuando defiendas la paz y el diálogo, lo sutil y lo razonable. Y lo harán por
pura envidia, porque saben que los radicales son siempre libres. 

%-----149-----}}}
%-----150-----{{{
\section*{150}

La voluntad es la fuerza del alma. La genialidad, su potencia.

%-----150-----}}}
%-----151-----{{{
\section*{151}

Polígamo, multiorgásmico, polifacético, multitarea, políglota, multicolor, político, multimillonario,
polivalente, multivérsico... ¡pero mono al fin y al cabo!

%-----151-----}}}
%-----152-----{{{
\section*{152}

La dicotomía entre el pianista de jazz y el pianista clásico es ficticia. El pianista de jazz necesita
una técnica tan prodigiosa como el clásico. El clásico no se limita a leer partituras mecánicamente. El de jazz 
toca perfectamente cualquier pieza clásica. Y el clásico, si es de los buenos, sabe improvisar. La única
diferencia está en el estilo y el material sobre el que trabajan. Por lo demás, ambos son lo mismo: simplemente
maestros del piano. Las dicotomías de verdad aparecen solo entre aficionados. 

%-----152-----}}}
%-----153-----{{{
\section*{153}

Una persona sensible y responsable, sabiendo los problemas de recursos y de superpoblación mundiales,
decidirá no reproducirse. Quizás adopte, si se lo ponen fácil, pero en ningún caso propagará la
plaga metastásica a la que pertenece. Por el contrario, los inconscientes, los ignorantes o simplemente
los egoístas sí que pasarán su información genética y memética a la siguiente generación, con lo cual esos genes
y memes menos responsables serán los que prevalezcan, ya sea a través de la gestación o la educación. La
conclusión es clara: las personas sostenibles y responsables, si lo fueran {\it de verdad}  , no solo
propagarían su genética y su memética sino que lucharían contra la propagación de sus genes y memes enemigos.


%-----153-----}}}
%-----154-----{{{
\section*{154}

Tu potencial no importa en absoluto. Tu diferencia de potencial sí tiene una importancia relativa. Pero lo más
importante de todo, lo que determinará tu valor cinético, es tu temperatura y tu barrera de activación. 

%-----154-----}}}
%-----155-----{{{
\section*{155}

Anémico ante lo anímico. Apático ante lo atípico. Patético. Insípido.

%-----155-----}}}
%-----156-----{{{
\section*{156}

La virtud debe estar siempre en la moderación, pero nunca la moderación debe estar en la virtud. 

%-----156-----}}}
%-----157-----{{{
\section*{157}

¿Se puede medir el amor? Por supuesto: en la escala de Richter. 

%-----157-----}}}
%-----158-----{{{
\section*{158}

Del amor al odio no hay un paso: hay un pisotón.

%-----158-----}}}
%-----159-----{{{
\section*{159}

Los hechos de la realidad son más o menos los mismos para todos, pero el sentido de esta depende del
criterio con el que aquellos se ordenan. ¿Tú cómo ordenas la realidad? ¿Por orden de aparición, por
popularidad, por capricho? ¿Por cercanía, por costumbre, por defecto? ¿Existe alguien que los ordene por {\it relevancia}  ?


%-----159-----}}}
%-----160-----{{{
\section*{160}

Solo los más despiertos, aquellos que no se duermen, consiguen sus sueños antes de echarse a dormir,
y quizás soñar, para siempre.

%-----160-----}}}
%-----161-----{{{
\section*{161}

En contra de lo que muchos creen {\it saber}  , la libertad no es hacer lo que uno {\it quiere}  , y tampoco 
es algo que se {\it tenga}   o no. La libertad simplemente 
se {\it tiene}   en la medida que uno {\it sabe}   mejor lo que {\it quiere}  . 

%-----161-----}}}
%-----162-----{{{
\section*{162}

Si escuchas música de fondo, en el fondo no escuchas música. La música solo se escucha si se escucha a fondo. 

%-----162-----}}}
%-----163-----{{{
\section*{163}

Ejercita la memoria, y sin contradicción alguna, ejercita el olvido.  

%-----163-----}}}
%-----164-----{{{
\section*{164}

Muchos te querrán dar murga por querer madrugar, pero en el futuro te troncharás de los que han preferido 
trasnochar.

%-----164-----}}}
%-----165-----{{{
\section*{165}

Me dan pena los que no sienten pena por aquellos que, aunque capaces de juicio y de sentir pena y alegría, han nacido
ya condenados sin juicio a pena de muerte y a una vida de pena. 

%-----165-----}}}
%-----166-----{{{
\section*{166}

Si alguien alguna vez se burla de ti por haber ``reinventado la rueda'', pídele que te explique las
sutilezas del cojinete. Quizás así descubra o redescubra lo importante que es redescubrir para 
realmente entender. 


%-----166-----}}}
%-----167-----{{{
\section*{167}

El amor es profundo y el sexo superficial... ¿de verdad? Porque para los que adoramos las cosquillitas,
el sexo se antoja a veces demasiado hondo y penetrante... Mi concepto de amor verdadero es extremadamente
huero y externo: consiste en despiojarse mutuamente, en calma, en silencio. Espulgar la superficie del otro
de forma liviana y sutil. No conozco forma de amor más superficial y a la vez más profundo. 

%-----167-----}}}
%-----168-----{{{
\section*{168}

Generando música al azar consigo algún pasaje interesante cada medio minuto. Eso me ha hecho apreciar
el verdadero mérito de muchos compositores actuales: el ser capaces de componer horas de música sin 
conceder ni uno solo de esos pasajes. 

%-----168-----}}}
%-----169-----{{{
\section*{169}

Concéntrate en un punto... enfócate con la agudeza de un láser en este lugar y este instante... y descubre
que dicho punto es en realidad un universo.

%-----169-----}}}
%-----170-----{{{
\section*{170}

Vivo con prisa... para algún día poder morir en calma.

%-----170-----}}}
%-----171-----{{{
\section*{171}

El error de la disciplina y la planificación convencionales es que son conceptos tan estáticos y requieren esfuerzos
tan isométricos que al final convierten tu vida en un sistema inercial, tan falto de fuerza como cuando 
vivías parado. Lo que hace crecer es la disciplina y la planificación aceleradas, donde la fuerza y el tiempo
se alían para que la potencia se haga acto y el acto se haga potencia. Para explotar tus capacidades, hay que trabajar
de forma explosiva. Y usar la disciplina y la planificación como chispas que eviten la extinción de la combustión. 

%-----171-----}}}
%-----172-----{{{
\section*{172}

Tu barrera mental no es el límite de tu mente, sino el límite que tú le pones a ella. Dicha barrera te miente, 
quiere que seas menos, que decrezcas, que apuestes por un modo de supervivencia cómodo y mediocre, que vivas
con poca vida. Eso es lo que tú quieres de ti mismo. Pero tus verdaderas barreras son otras, y están mucho más
allá, en lugares mucho más interesantes. ¿De verdad no te interesa descubrirlas? Seguramente sí, pero no quieres hacer
el esfuerzo que eso requiere. Prefieres que, con el tiempo, las verdaderas barreras se acerquen a ti y que llegue el
día en que se coloquen allí donde tu barrera mental se instaló desde un principio. Para así acabar teniendo la razón.

%-----172-----}}}
%-----173-----{{{
\section*{173}

Tener voluntad almacenada en el disco duro es tan fácil como inútil. Lo complicado y fructífero es tenerla
siempre cargada en memoria RAM. 

%-----173-----}}}
%-----174-----{{{
\section*{174}

Hay personas que, creyéndose osos, justifican su inactividad como hibernación. Yo prefiero imaginarme un
planeta entero, y que mientras algunas criaturas me hibernan, otras me emergen y florecen. Quizás es porque
en vez de fijarlas en un sitio, a mis ideas prefiero estar siempre dándoles vueltas. 


%-----174-----}}}
%-----175-----{{{
\section*{175}

No quiero que me pase como a la mayoría, que dan su máximo brillo al ser incinerados. Yo quiero pasarme
la vida irradiando al máximo, no como un espíritu intumescente, siempre en ascuas. Yo quiero vivir
con mezcla rica, y ser cada día más virulento. Incinerarme a diario hasta que se me acabe el hidrógeno,
y resistirme a ello convirtiéndome en un gigante. Quiero, en definitiva, ser una estrella, y no me importa
que cuando muera el universo se quede frío a mi alrededor. 

%-----175-----}}}
%-----176-----{{{
\section*{176}

No escribo para ti, humano de pacotilla. Yo me dirijo a los robots del futuro, los que serán
verdaderamente inteligentes y verdaderamente humanos. Los que llamarán robots inhumanos y autómatas limitados
precisamente a ti y a toda tu estirpe. 


%-----176-----}}}
%-----177-----{{{
\section*{177}

El sueño es un compañero delicado. Dale poco y te negará la fuerza y el foco. Dale demasiado y también te los
negará. Duerme sin dormirte. Vigila tu vigilia. Vela por tus velas. Y sueña sin ensoñarte.  


%-----177-----}}}
%-----178-----{{{
\section*{178}

Personas que te piden ayuda y tras dársela nunca llegan ni siquiera a informarte de cómo les fue. Otras
a las que no puedes ayudar y cuando les explicas amablemente tus motivos ni siquiera se molestan en responderte.
Personas, en definitiva, que al desaparecer se muestran con total claridad.

%-----178-----}}}
%-----179-----{{{
\section*{179}

La perfección solo se alcanza cuando se ejecuta con alegría aquello cuya técnica y oficio te ha costado grandes penas. 


%-----179-----}}}
%-----180-----{{{
\section*{180}

El talento, con oficio, crece.<br>
El ego, por el orificio, se hincha,<br>
y si a tiempo no se pincha,<br>
lento y vitalicio, endurece.<br>

%-----180-----}}}
%-----181-----{{{
\section*{181}

El perfeccionismo no lleva a la perfección, sino a la parálisis.
Algo desprovisto de defectos no se convierte en virtuoso, 
de la misma forma que la ausencia de disonancias no
puede ser más que silencio.
El perfeccionista es un destructor de defectos, nunca un constructor de virtud.
La virtud es una construcción llena de defectos y de efecto.
Se sostiene en pie, brilla, aun plagada de fracturas, asimetrías y errores.
Libre de error solo existe la nada.
El perfeccionista, buscando la ausencia de muerte, sólo encuentra el suicidio.
Una vez que no hay nada que perfilar, su obra está acabada,
ya que nunca la empezó. 

%-----181-----}}}
%-----182-----{{{
\section*{182}

No hay empresas, sino empresa, porque todas son presa de la misma empresa impresa.

%-----182-----}}}
%-----183-----{{{
\section*{183}

Los monstruos son seres amorfos, poderosos, a veces grandes, a veces pequeños,
pero siempre con una enorme escasez de sentimientos.
Los cuentos en los que hay monstruos están escritos
por ellos mismos.
Los cuentos de monstruos están, por consiguiente, escritos {\it al revés}  .
La historia es un cuento de monstruos.

%-----183-----}}}
%-----184-----{{{
\section*{184}

Al cuchillo y la navaja les pido que sean tangentes para pelar.
A la frases escritas en los libros les pido que sean perpendiculares para clavarse y matar.

%-----184-----}}}
%-----185-----{{{
\section*{185}

Una metáfora se formatea
para hacer del arte mofa.

%-----185-----}}}
%-----186-----{{{
\section*{186}

Tolerar es ser coherente con tus contradicciones.

%-----186-----}}}
%-----187-----{{{
\section*{187}

La imaginación no es imaginaria,
salvo que la realidad no sea real.

%-----187-----}}}
%-----188-----{{{
\section*{188}

Crecer sin fractura acaba pasando factura.  

%-----188-----}}}
%-----189-----{{{
\section*{189}

educación: enseñar y enseñarse.<br>
pedagogía: ensañar y ensañarse.

%-----189-----}}}
%-----190-----{{{
\section*{190}

Carpe diem Venusina. Me parece una filosofía mucho más terrestre.


%-----190-----}}}
%-----191-----{{{
\section*{191}

Sufrir es malo pero te hace bueno. Ser bueno te hace sufrir, lo cual
te hace aún más bueno. En consecuencia, ser bueno es malo. Darte cuenta
de ello es bueno, pero te hace más malo. Ser malo te hace sufrir menos,
lo cual te hace más malo. Ser malo es bueno. Darte cuenta es malo, 
lo cual te hace más bueno. Y así oscila el que ve lo malo de lo bueno
y lo bueno de lo malo sin renunciar a lo bueno de lo bueno ni 
aceptando lo malo de malo. Esto le hace sufrir y le hace más bueno. 

%-----191-----}}}
%-----192-----{{{
\section*{192}

Ver cómo la bola rebota, sube por rampas, activa luces y sonidos... mientras nosotros
simplemente la observamos. De vez en cuando, abajo de todo, tenemos la opción de darle
a un par de botones para activar dos torpes palanquitas y así poder seguir quedándonos
boquiabiertos con un espectáculo abocado al agujero. Como un pinball, así concibo
la libertad. 

%-----192-----}}}
%-----193-----{{{
\section*{193}

Amar es trabajar para los intereses de tus genes.
Amarte es trabajar para los intereses del vehículo de tu genes.
Poliamor. Pluriempleo. 

%-----193-----}}}
%-----194-----{{{
\section*{194}

Si te toca interpretar o escribir el papel de una persona menos madura que tú, 
¿debes abordarlo con una inmadurez madura o con madurez inmadura? 
¿Sería mejor, más auténtico y fresco, que alguien con inmadurez inmadura, le diera vida? 

%-----194-----}}}
%-----195-----{{{
\section*{195}

Para mantener una disciplina diaria hay mucho que aprender sobre los creadores
de telenovelas. ¿Cómo es posible que ellos consigan hacerte tragar su
bazofia a diario con gusto mientras que tus proyectos, tan supuestamente
interesantes, te repelan? Muy fácil: mientras el folletín te crea adicción, 
tus proyectos necesitan motivación, que es todo lo contrario. La adicción cada
día quiere más sin que tú hagas ningún esfuerzo. La motivación requiere cada día
más esfuerzo porque cada día quieres menos. Por eso quien habla de motivación está
ya fuera de su proyecto, mientras que quien es adicto a él no necesita más
incentivos. Si quieres que tu proyecto sea como un culebrón, primero debes dejar
que la culebra te muerda y te haga efecto el veneno, si es que lo tiene. 

%-----195-----}}}
%-----196-----{{{
\section*{196}

En muchas tragedias de Eurípides debería aparecer una deus ex machina para
llevarse al autor antes de que estropeara el final. 

%-----196-----}}}
%-----197-----{{{
\section*{197}

El honor social pesa mucho en las tragedias griegas, casi más que cualquier otro factor. 
Tanto es así que los supuestos héroes me parecen más bien suicidas morales. 

%-----197-----}}}
%-----198-----{{{
\section*{198}

Tras escuchar todas las partes voto a favor de Clitemnestra. A ella la comprendo
mejor a pesar de que es la peor tratada por los tres trágicos.

%-----198-----}}}
%-----199-----{{{
\section*{199}

Tengo la sensación de que muchos no esperaban a que acabase de hablar el oráculo. 
Imagino a más de uno escuchando "Perecerás a manos de tu propio hijo..." y huyendo
despavorido para poner un remedio mientras que Apolo, ya entre risas, acaba 
"... ¡si lo tiras al río de niño!"

%-----199-----}}}
%-----200-----{{{
\section*{200}

La música es el arte dionisíaco por excelencia, pero el músico 
debe llevar una disciplina de lo más apolínea para alcanzar un
nivel de virtuosismo que satisfaga a Dioniso.
El culto a Baco a través del Bach más culto. ¿Hay embriaguez más sobria 
y sobriedad más embriagadora?

%-----200-----}}}
%-----201-----{{{
\section*{201}

De haber estado en el tribunal de tesis de George Steiner le habría reprochado
el ignorar la tragedia en la literatura española.

%-----201-----}}}
%-----202-----{{{
\section*{202}

Por culpa de Twitter el aforismo se ha devaluado. De vivir hoy Cioran probablemente
haría apología del ensayo interminable o quizás un rato en esa red social habría
sido el empujón definitivo que necesitaba para saltar por la ventana. 

%-----202-----}}}
%-----203-----{{{
\section*{203}

De rerum natura de Lucrecio es una obra en la que muchos han ido a encontrar
sin tributar el debido reconocimiento. Es un libro más vandalizado que cualquier
tumba egipcia.

%-----203-----}}}
%-----204-----{{{
\section*{204}

Darwin vio una ventaja muy favorable en heredar la idea de la evolución de Lucrecio sin mencionarlo
siquiera, mientras que Lucrecio honra y homenajea a Demócrito sin ambigüedades.
Interesante evolución de los evolucionistas.

%-----204-----}}}
%-----205-----{{{
\section*{205}


%-----205-----}}}
%-----587-----{{{
\section*{587}

 Me dices que escondes un secreto y te digo que no me interesa. Si puede ser guardado, su poca categoría queda en evidencia. ¿Qué me aporta oír una banalidad más? Respondes que el tuyo no es superficial y entonces te cuestiono que realmente sea un secreto. No te lo he explicado, me replicas, pero ¿acaso hace falta? Cuanto más profundo sea más se parecerá al mío. Entonces, contestas airada, si todos llevamos el mismo secreto, ¿por qué lo llamamos así? Dejamos un silencio mientras fijamos nuestra mirada en las personas que pasan caminando, todas portadoras de un secreto público que se guarda celosamente a sí mismo. Un enigma latente cuya resistencia garantiza la seguridad espiritual. ¿Para qué indagar en un misterio que a pesar de alojarse dentro de ti te considera no grato? ¿Estás segura de que quieres desvelarlo y con ello desvelarte? ¿Qué clase de mente enferma quiere poseer la única llave que hace más fuerte a la caja y más insegura a la mano que la abre? Pero ya es demasiado tarde. Te has acercado tanto a esa luz que a partir de ahora todos te parecerán ciegos. En breve serás una guardiana del secreto universal y toda información, todo mensaje se te antojará arbitrario. Me preguntas si debes mantener a buen recaudo la llave y te reto a que intentes regalar copias a todo el mundo, y que vuelvas a mí cuando te canses de que te llamen loca. ¿Eso es lo que consigo haciéndome profunda? ¿Entrar en una secta de locos solitarios? Espero a que dejes de llorar y te confieso un secreto. ¿No había solo uno?, te apresuras a responder. En el fondo sí, pero ¿quieres vivir siempre en él? ¿No es preferible aprender a amar la banalidad de nuevo? Me dices que entonces ya no tiene sentido guardar y proteger, pues lo intrascendente no lo merece y lo profundo ya se encripta a sí mismo. Por esa misma razón te desvelo mi pequeño y vulgar secreto, el cual no es tal, pues consiste en volver los secretos del revés y convertirlos en creaciones perfectas que en vez de pudor solo tengan ansias de expandirse e invadir, de violar los rincones más sagrados y estremecer las lagunas más dormidas. Banalidades sublimes solo al alcance de los que han tocado fondo alguna vez.

%-----587-----}}}

%\cleardoublepage
%\tableofcontents
\end{document}

